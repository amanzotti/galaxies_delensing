
\documentclass[12pt]{amsart}
\usepackage[top=2.5 cm, bottom=2.5 cm, left=1.05in, right=1.05in]{geometry}
%\geometry{a4paper} % or letter or a5paper or ... etc
% \geometry{landscape} % rotated page geometry
\usepackage[applemac]{inputenc}
\usepackage{amsmath,amssymb} 
% See the ``Article customise'' template for come common customisations

\title{Referee Response}
\author{A. Manzotti}
\date{} % delete this line to display the current date

%%% BEGIN DOCUMENT
\begin{document}

\maketitle

\begin{enumerate}
\item The abstract is unnecessarily very long and could be considerably
shortened. 

\vspace{0.4cm}
The abstract length is now reduced to $~80\%$ of its previous length, mainly removing redundancy and quantitative results that can be found easily in the tables but leaving the main general points.
\vspace{1cm}

\item Putting the lmin-lmax ranges in the tables as well as having the
dashed grey line of Fig. 5 onto Fig 6-7-8 would help the readability
of the paper. 
\vspace{0.4cm}

I agree with the referee on this point. I first thought this can cause a bit of clatter, but the gain in readability is worth the risk.
I added the dashed line to all the figures.

\vspace{1cm}
\item Twice in the paper the author writes that lensing is only one of
secondaries producing B-modes, or his writing gives that impression.
It is unclear to me what is meant by that, as far as I know, galactic
foregrounds should be basically the only expected source of worries
for the experiments considered there. 

\vspace{0.4cm}

We agree that the phrasing of the sentences where secondary B-modes sources are introduced can be confusing. While it is well known in the field, it is definitely important to be clear to a wider audience.
For this reason we add this sentence in the introduction:

\textit{Two main effects produce non-primordial B-modes: the polarized foregrounds from our own galaxy  and the effect of gravitational interactions with large scale structures (LSS). 
This two components needs to be treated with very different techniques and in this work we will focus on the latter (see \cite{remazeilles:2017b} for a review on the former).}

\vspace{1cm}
\item The author introduces sharp cuts to CIB and WISE at l < 100. Is the
alternative approach of incorporating the foreground content in the
auto spectra of the tracer (E.g. Eq. 17) combination possible ? That
would retain some more of the signal presumably. 
\vspace{0.4cm}

\vspace{1cm}
\item Fig 1 might look confusing at first. The SKA kernel there shows
brilliant overlap with the CMB lensing kernel, but is not the most
powerful tracer. On the other hand, DES for instance has very moderate
overlap but can still achieve very high performance on large scales.
Can the author comment and help the reader make sense of that figure ? 

\vspace{0.4cm}

\vspace{1cm}

\item Is it clear, as the author claims in IV A with no supporting
evidence, that fixing the cosmology has no impact on the {r,nt}
constraints ? The cosmology fixes the residual B-mode amplitude, which
is unequally important in the delensed vs nominal cases. An slightly
improved discussion would be welcome here. 

\vspace{0.4cm}

\vspace{1cm}


\item The author makes a strong point that survey data should be split
into bins, as this makes the correlation increases by 30%. I am not
entirely convinced that this is as relevant as claimed. From Fig 4, it
seems that the improvement is fairly limited on the scales that are
relevant for B-mode delensing. Presumably, that is more important for
T or E delensing. The author might want to comment or adapt. 

\vspace{0.4cm}

\vspace{1cm}
 
\end{enumerate}



\end{document}