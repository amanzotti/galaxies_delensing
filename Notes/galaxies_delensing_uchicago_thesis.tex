% ---- ETD Document Class and Useful Packages ---- %
\documentclass{ucetd}
\usepackage{subfigure,epsfig,amsfonts}
\usepackage{natbib}
\usepackage{amsmath}
\usepackage{amssymb}
\usepackage{amsthm}
\usepackage{natbib}
\usepackage{enumitem}
%\usepackage{hyperref}
\input{./aas_abbr.tex}
\usepackage{bm}
\usepackage{epsfig}
\usepackage{natbib}
\usepackage{graphicx,epsfig}
\usepackage{xstring}
\newcommand{\relook}[1]{{\textcolor{black}{\bf #1}}}
\usepackage[justification=centering]{caption}

% \usepackage{graphicx}\usepackage[top=2.5 cm, bottom=2.5 cm, left=1.05in, right=1.05in]{geometry}
%\usepackage[applemac]{inputenc}
%\usepackage{amsmath,amssymb}
\usepackage{color}
%\usepackage{epstopdf}

\newcommand{\al}[1]{\begin{align} #1 \end{align}}

\newcommand{\note}[1]{\textsc{#1}}
	\newcommand{\bR}[1]{{\bm {\mR{#1}}} }

\def\l{\ell}

\newcommand{\nver}{\hat{\mathbf{n}}}
\newcommand{\cov}{\text{Cov}}
\newcommand{\Nsim}{N_{\text{sim}}}
\newcommand{\new}[1]{{\textcolor{blue}{\bf #1}}}
%--------- NEW COMMAND STUFF
	\newcommand{\PD}[2]{\dfrac{\partial #1}{\partial #2}}

\def\simlt{\lesssim}
\def\simgt{\gtrsim}
\newcommand{\Omegamzero}{\Omega_{{\rm m,0}}}
\newcommand{\alm}{a_{lm}}
	% * subscript
	\def\rom#1{%
		_{\mathrm{#1}}%
	}%
%\newcommand{\l}{\ell}}

\def\be{\begin{equation}}
\def\ee{\end{equation}}
\def\ben{\begin{equation*}}
\def\een{\end{equation*}}

\def\ba{\begin{eqnarray}}
\def\ea{\end{eqnarray}}
\def\ban{\begin{eqnarray*}}
\def\ean{\end{eqnarray*}}

\newcommand{\refsec}[1]{section~\ref{sec:#1}}
\newcommand{\reftab}[1]{Tab.~\ref{tab:#1}}
\newcommand{\refeq}[1]{Eq.~(\ref{eqn:#1})}
\newcommand{\refssec}[1]{section~\ref{subsec:#1}}
\newcommand{\reffig}[1]{Fig.~\ref{fig:#1}}
\newcommand{\refFig}[1]{Fig.~\ref{fig:#1}}
	\newcommand{\mR}[1]{\mathrm{#1}}   % mathrm



\definecolor{darkgreen}{cmyk}{0.85,0.2,1.00,0.2}
\definecolor{purple}{cmyk}{0.5,1.0,0,0}


\newcommand{\bx}{{\bf x}}
\newcommand{\bk}{{\bf k}}
\newcommand{\bq}{{\bf q}}
\newcommand{\bP}{{\bf\Psi}}
\newcommand{\bs}{{\bf s}}
\newcommand{\cs}{{\cal S}}
\newcommand{\by}{{\bf y}}
\newcommand{\deltar}{\delta_{\rm recon}}%\def\lcdm{$\Lambda$CDM}
\def\arcsec{$^{\prime\prime}$}
\def\nl{N_\ell}
\def\sl{S_\ell}
\def\ublu{\bl'}
\newcommand{\conv}[2]{\left(\frac{#1}{#2}\right)}
\def\arctanh{\mathop{\rm arctanh}\nolimits}
%\renewcommand{\eqref}[1] {equation $($\ref{#1}$)$}
\newcommand{\comment}[1]{{}}
\def\beq{\begin{equation}}
\def\eeq{\end{equation}}
\def\beqn{\begin{eqnarray}}
\def\eeqn{\end{eqnarray}}
\def\a{\alpha}
\def\cl{C_{l}}
\def\h{\mathrm{h}}
\def\d{\rmn{d}}
\def\pa{\partial}
\def\deldel#1#2{\frac{\pa{#1}}{\pa{#2}}}
\def\ba{\bm{\alpha}}
\def\fracj#1#2{{\textstyle{#1\over#2}}}
\def\bxi{\bm{\xi}}
\def\half{\frac{1}{2}}
\def\ti{\widetilde}
\def\O{\Omega}
\def\OL{\Omega_\Lambda}
\def\Om{\ensuremath{\Omega_{\mathrm{m}}}}
\def\Ob{\ensuremath{\Omega_{\mathrm{b}}}}
\def\Oc{\ensuremath{\Omega_{\mathrm{CDM}}}}
\def\msol{\ensuremath{M_\odot}}
\def\l{\left}
\def\r{\right}
\def\o{\omega}
\def\gcm{\textrm{g cm$^{-3}$}}
\def\2gcm{\textrm{g cm$^{-2}$}}
\def\Scr{\Sigma_{\mathrm{crit}}}
\def\rcr{\rho_{\mathrm{crit}}}
\def\phidot{\ensuremath{\dot\phi_{\bl,\bl'}}}
\def\ddelta{\ensuremath{\dot\delta}}
\def\modu#1{\l |{#1}\r |}
\def\av#1{\l \langle{#1}\r \rangle}
\def\hmpc{\:{h}^{-1}\mathrm{Mpc}}
\def\th{\Theta}
\def\tth{\tilde\Theta}
\def\sg{\sigma}
\def\Sig{\Sigma}
\def\cf{{\cal F}}
\def\k{\kappa}
\def\P{{P}}
\def\pnl{{P}_{{\!\textrm{\tiny NL}}}}
\def\dnl{\Delta_{\text{\scriptsize NL}}}
\def\kmin{\k_{\mathrm{min}}}
\def\kmax{\k_{\mathrm{max}}}
\def\hires{the \emph{highRes} experiment}
\def\lores{the \emph{lowRes} experiment}
\def\ktot{\k_{\mathrm{tot}}}
\def\hunit{\ensuremath{\mathrm{km}{\mathrm{s}^{-1}} \mathrm{Mpc}^{-1}}}
\def\H0{\ensuremath{\mathrm{H}_0}}
\def\nn{\nonumber}
\def\lin{\mathrm{lin}}
\def\ISW{\mathrm{ISW}}
\def\bl{\bmm{l}}
\def\bL{\bmm{L}}
\def\fsky{f_{\mathrm{sky}}}
\newcommand{\E}[1]{\times 10^{#1}}
\newcommand{\bmm}[1]{{\mathbf{#1}}}
\newcommand{\bsection}[1]{\chapter{\uppercase{#1}}}

%% Use these commands to set biographic information for the title page:
\title{Future Cosmic Microwave Background delensing with galaxies surveys}
\author{Alessandro Manzotti}
\department{Astronomy and Astrophysics}
\division{Physical Sciences}
\degree{Doctor of Phylosophy}
\date{August 2017}
\dedication{Dedication Text}
\epigraph{No epigraph}


\begin{document}

\maketitle
\makecopyright
%\makededication
%\makeepigraph
\tableofcontents
\listoffigures
\listoftables
        
\abstract

The cosmic microwave background (CMB) polarization is a promising experimental dataset to test the inflationary paradigm and to probe the physics of the early universe. 
A particular component, the so-called B-modes, is indeed a direct signature of a prediction of inflation: the presence of gravitational waves in the early universe. However, improving the level of noise in future experiments will not be enough. 
Secondary effects in the low redshift universe will also produce non-primordial B-modes adding confusion to the inflationary signal. In particular, the gravitational interactions of CMB photons with large scale structures will distort the primordial E-modes adding a lensing B-mode component to the primordial signal.
Removing the lensing component from the measurement of CMB B-modes will then be necessary to constrain the amplitude of the primordial gravitational waves.
Here we discuss the role of current and future large scale structure surveys in multi-tracers approach that will improve the reconstruction of the lensing potential that lenses the CMB photons and and, as a consequence, the delensing efficiency. 
We quantify this by the improvement due to delensing on the constraints on the inflationary tensor perturbations amplitude and shape ($r$ and $n_{t}$).
We find that a galaxy survey like DES can remove about $14\%$ of the lensing signal. current generation CMB experiments will benefit from 
With the improvement of the noise CMB internal reconstruction will become more important and the fraction of removed lensing B-modes will rapidly improve from the current level of Planck ($8\%$) and SPTPol ($35\%$) to 3G (56\%) and CMB S4 (76\%) level. 
Nevertheless optical galaxy surveys, in particular if split in different tomographic bin will still play an important role even for CMB S4. 
\mainmatter


\chapter{Introduction}
\label{sec:intro}
In the standard cosmological paradigm the early universe underwent a period of near-exponential expansion called "cosmic inflation." 
All the cosmological observations agree with this picture, making it a compelling and elegant description of the Universe initial conditions. Despite the experimental effort, other possible explanations are still valid, and a conclusive evidence of inflation is still to be found. 
Inflation generically predicts a stochastic background of gravitational waves \cite[see e.g.][for a review]{kamionkowski15}. This prediction set inflation apart from other theories and a detection of primordial gravitational waves could be the compelling evidence cosmologists are looking for. 
These primordial gravitational waves in the early universe would imprint a unique signature on the polarized anisotropies of the CMB. For this reason, CMB polarization is a promising dataset to understand the physics of the early universe and ultimately test inflation.
In particular, we can decompose the CMB polarization fields in Fourier space into even-parity (divergence) and odd-parity (curl) components, referred to as ``E'' and ``B'' modes.
 In the standard scenario, the B-mode polarization is a clean probe of primordial gravitational waves, because these are the only source of B-modes at the epoch of recombination.

Because the CMB B-modes provide the cleanest known observational window into the primordial gravitational waves background, improving their measurement is a major objective of current and future CMB experiments.
Even if inflationary B-modes have not been detected yet, a natural value $r \gtrsim 10^{-3}$ should be reachable shortly given the level of noise expected in future CMB experiments \citep{kamionkowski15}.
However, just reducing the level of noise will not be enough to attain this goal.
Indeed, the observed B modes are not solely sourced by early universe physics; they are also produced by secondary effects taking place in the late-time low-redshift universe. 
In this work, we will focus on the effect of gravitational interactions with large scale structures (LSS).
Lensing shears the CMB polarization pattern, producing ``lensing B modes'' from CMB E modes \citep{zaldarriaga98}.
This expected component has now been measured both in cross-correlation with LSS \citep{hanson13,polarbear2014c,vanengelen14b,planck2015XV} and from CMB data alone \citep{polarbear2014b,bicep2a,keisler15}.

This component acts as a source of confusion for searches of the primordial gravitational wave background.
Indeed, the contamination from lensing B modes is already at the level of the instrumental noise of current experiments \citep{bk14}. Thus, together with the experimental effort to reduce the amount of noise, the effect of the spurious lensing component must be understood.

The optimal way to reduce the lensing contributions to the B-modes is to reconstruct the expected realization of lensing B modes on the observed part of the sky and then use it to clean the data in a process called ``delensing''. 
We can delens the observed B-modes by combining CMB polarization data (what is \textit{lensed}) with tracers of the large scale structure (what is \textit{lensing}) to reconstruct a template of the expected lensing B-modes.
Delensing has been studied for many years \cite{knox2002, kesden2002, seljak2003, simard:2015,sherwin15,smith:2012}.
Furthermore, it has recently been performed on CMB temperature data using the cosmic infrared background as LSS tracer \cite{larsen:2016} and on CMB temperature and polarization data using CMB data to internally reconstruct the LSS lensing potential \cite{ carron17}. Finally, the highest B-mode delensing efficiency has been achieved with SPT and Herschel data in which $~28\%$ of the lensing power was removed.

For future experiments, we need to increase the delensing efficiency by almost a factor of 3 to fully exploit the expected instrumental capabilities \cite{abazajian:2016}.
In this paper, we propose and study a possible way: using future galaxy surveys as tracers of the lensing potential in addition to other probes such as the SKA radio continuum survey, cosmic infrared background and internal CMB reconstruction. 
Furthermore, we point out how using redshift information through tomographic binning can improve the delensing efficiency of galaxy survey.
These will translate into a better reconstruction of the B-modes in the measured patch, and, as a consequence will improve the constraints on inflation through delensing. We model several actual and future surveys, and after computing the residual B-modes, we forecast the resulting statistical uncertainties on the amplitude and the shape of the inflationary tensor perturbations for current CMB experiments as well as the next generation (S3) and the planned fourth generation (S4). 

We organize this article as follow: we describe the LSS tracers used in this analysis in \refsec{model}. In \refsec{th} we define the lensing B-mode component and the residual power after delensing with tracers of the lensing potential. The main result of this work is \refsec{for} : the improvement of inflationary parameters constraints due to delensing with CMB and LSS. We conclude in \refsec{concl}.



\chapter{Lensing potential tracers}
\label{sec:model}
In this section, we introduce the different large scale structure tracers considered in this work to reconstruct the lensing potential. Gravitational lensing distorts primordial E-modes generating a non-primordial B-mode component.
Also, we define the power spectra that we will use later in \refsec{th}. 
 

Large scale structure surveys usually probe the 3D dark matter overdensities as a 2D field projected along the line of sight:

\begin{equation}
\delta^{i}(\nver) = \int_0^{\infty} dz\, W^{i}(z)\delta(\chi(z)\nver,z).
\label{eqn:wkernel}
\end{equation}

where $\delta(\chi(z)\nver,z)$ corresponds to the dark matter overdensity field at a comoving distance $\chi(z)$ and at a redshift $z$ in the angular direction $\nver$.
Using the Limber approximation \cite{limber53} we can compute the power spectra of two large-scale structure fields $i,j$ as:
\begin{equation}
C_{\ell}^{ij}= \int_0^{\infty} \frac{dz}{c} \frac{H(z)}{\chi(z)^2}\, W^{i}(z)W^{j}(z)P(k,z).
\label{eqn:wkappa}{}
\end{equation}

In this equation, $H(z)$ is the Hubble factor at redshift $z$, $c$ is the speed of light and $P(k,z)$ is the matter power spectrum evaluated at wavenumber $k = \ell/\chi(z)$ and redshift $z$. 
Furthermore $W^{i}(z)$ is the kernel function of the field $i$ introduced in \refeq{wkernel}.
We will now describe the kernels for each of the tracers used in this work.


\section{CMB lensing potential}
\label{sec:kappaCMB}
We start from the CMB lensing potential.
The lensing kernel $W^{\kappa}$ is:
\begin{equation}
W^{\kappa}(z) = \frac{3\Omega_{\rm m}}{2c}\frac{H_0^2}{H(z)}(1+z)\chi(z)\frac{\chi_*-\chi(z)}{\chi_*},
\end{equation}
%
where $\chi_*$ is the comoving distance to the last-scattering surface at $z_*\simeq 1090$, $\Omega_{\rm m}$ and $H_0$ are the present day values of the Hubble and matter density parameters, respectively.

The CMB lensing potential is \textit{the} field that we need to reconstruct to reverse the effect of large scale structure and delens the CMB.
However, the lensing potential can also be reconstructed using the CMB itself. In that case, we can treat it as a noisy tracer of the true field. Both the CMB lensing field and its noisy reconstructed counterpart have the same kernel $W^{\kappa}(z)$.
However, when computing the power spectrum of the latter, we need to add a noise component. Given the instrumental noise level and the beam, we can calculate the reconstruction noise $N_{\ell}^{\kappa \kappa}$, and so
\beq\label{eqn:noisekappa}
\cl^{\kappa_\mathrm{rec} \kappa_\mathrm{rec}} = \cl^{\kappa \kappa } +N^{\kappa \kappa}_{\ell}
\eeq
In this work, the level of noise is computed assuming an iterative approach to the CMB lensing reconstruction as described in \cite{smith:2012,hirata:2003}.



\section{Galaxies}

The galaxy clustering kernel is:
\begin{equation}
\label{eqn:wg}
\begin{split}
W^{g}(z) &= \frac{b(z)\frac{dN}{dz}}{\Bigl(\int dz'\,\frac{dN}{dz'}\Bigr)}.
\end{split}
\end{equation}
Here $\frac{dN}{dz}$ is the number of galaxies observed by the survey as a function of redshift while $b(z)$ is the galaxy bias that connects the amplitude of galaxy overdensities to the underlying dark matter density.
When computing the auto-spectrum of the galaxy density, a shot noise term needs to be taken into account. 
To do so, we add a constant term to the power spectrum equal to the inverse of the number of galaxies per steradians.
 Different galaxy surveys in this work are then fully characterized by their b(z), $\frac{dN}{dz}$ and the observed galaxy density.
We test the delensing efficiency taking into account both current surveys like WISE or DES as well as future galaxy surveys like DESI and LSST together with 21 cm measurement like SKA.


The WISE survey observed the entire sky in the infrared \cite{wright:2010}.
We defined the redshift distribution of the WISE infrared galaxy samples following \cite{yan:2013} (see Fig. 4 therein). 
To compute the noise term, we assume that the available sky after masking is around $f_{sky}=0.44$ with 50 million galaxies \cite{ferraro:2015} and that the galaxy density is approximately uniform. Furthermore we adopt a linear bias $b_{\rm{WISE}}=1.41$ obtained by \cite{ferraro:2015} cross correlating WISE with Planck lensing potential.

DES is modeled after the DES Science Verification public data release.
For DESI we used the $\frac{dN}{dz}$ in Tab. 2.3 of the DESI Technical Design Report. From that we can derive the galaxy denisty of 0.63 galaxies per arcmin squared. 
%As in \cite{passaglia:2017} we use a linear bias $b$

For LSST we follow \cite{schaan:2016}: $\frac{dN}{dz}\propto z^{\alpha}\exp^{-(z/z_{0})^{\beta}}$ with $\alpha = 1.27$, $\beta = 1.02$, and $z_{0}= 0.5$. Furthermore we assume a density of 26 galaxies per arcmin squared.

Finally, we consider the Square Kilometre Array (SKA).
The SKA is a planned radio array that will survey large scale structure primarily by detecting the redshifted neutral hydrogen (HI) 21cm emission line from a large number of galaxies out to high redshift. 
We will assume an intensity mapping survey mode where SKA will measure the large-scale fluctuations of the \textit{integrated} 21cm intensity from many unresolved galaxies.  
We model both the redshift distribution and  bias of radio sources following \cite{namikawa:2016a}. 

%bias for the Radio Continuum (RC) survey conducted by SKA

\section{Cosmic infrared Background (CIB)}
\label{sec:cib}

The CIB consists of diffuse extragalactic radiation generated by the unresolved emission from star-forming galaxies (see \cite{dole:2006} and references therein). 
In these galaxies, the UV light from young stars heats the dust regions around them that then reradiates thermally in the infrared with a graybody spectrum of $T \simeq30$K. 

Following \cite{Addison:2011se}, we model the CIB power directly as $C_{\ell}^{{\rm CIB}\mbox{-}{\rm CIB}} = 3500 (l/3000)^{-1.25} {\rm Jy^2 / sr}$.
This model provides an accurate fit to several experimental results.
For the cross-spectra with the CMB lensing or other galaxy tracers, $C_{\ell}^{{\rm CIB}\mbox{-}j}$, we use the single-SED model of~\cite{hall10}.
It corresponds to the kernel:
\begin{align}\label{eq:12}
W^{\text{CIB}}(z) = b_c\ \frac{\chi^2(z)}{H(z)(1+z)^2}\ e^{-\frac{(z-z_c)^2}{2\sigma^2_z} } f_{\nu(1+z)},
\end {align}
for
\begin{equation}
f_{\nu} = 
\begin{cases}
\Big( e^{\frac{h\nu}{kT}} - 1 \Big)^{-1} \nu^{\beta+3} & (\nu \leq v^{\prime}) \\ \Big( e^{\frac{h\nu^{\prime}}{kT}} - 1 \Big)^{-1} \nu^{\prime \beta+3} \Big( \frac{\nu}{\nu^{\prime}} \Big)^{-\alpha} & (\nu > v^{\prime})
\end{cases}
\end{equation}

We place the peak of the CIB emissivity at redshift $z_c = 2$ with a broad redshift kernel of width $\sigma_z = 2$ and we set $T$ = 34K and $\nu^{\prime} \approx$ 4955 GHz.


\chapter{Gravitational lensing B-mode and delensing}
\label{sec:th}
The large scale structures described in \refsec{model} have an important impact on the search of primordial CMB B-modes: they lens the primordial E-modes generating non inflationary B-modes that constitues an important source of noise.

Indeed the Q and U mode decompositions of the CMB photons polarization are remapped by lensing as:
\be
Q(\hat{\mathbf{n}}) = Q_{\mathrm{unlensed}}(\hat{\mathbf{n}}+\nabla\phi);~~
U(\hat{\mathbf{n}}) = U_{\mathrm{unlensed}}(\hat{\mathbf{n}}+\nabla\phi)
\ee
where the deflection angle is the gradient of the lensing potential integrated along the line of sight $\nabla\phi$.
The CMB polarization is usually decomposed into odd-parity Fourier modes E and B. 
As shown in \cite{}, because of the symmetry of the problem, tensor perturbations are the principal source of the B-modes configuration. For this reason, B-modes are a promising signature of early universe tensor perturbations.

However, this promising sign of primordial gravitational waves is partially obscured by gravitational interactions with large scale structures that generate CMB B-modes by distorting primordial E-modes.
At first order, given the convergence field $\kappa= -\frac{1}{2}\nabla^2\phi$ introduced in \refsec{kappaCMB} the B-modes resulting from the lensing of primordial E-modes are:
\be
B^{\mathrm{lens}}(\bl) =  \int \frac{d^2 \bl'}{(2 \pi)^2} W(\bl,\bl') E(\ublu) \kappa(\bl - \bl')
\label{eqn:blens}
\ee
where different modes contributes with a different weight:
\be
W(\bl,\bl') = \frac{2 \bl' \cdot (\bl-\bl')}{|\bl-\bl'|^2} \sin(2\varphi_{\bl,\bl'}).
\ee
%As usual we define the power spectrum as:
%\be
%\langle B^{\mathrm{lens}}(\bl) ~ B^{\mathrm{lens}^*}(\tilde{\bl}) \rangle \equiv (2\pi)^2 \delta^D(\bl - \tilde{\bl}) C^{BB,\mathrm{lens}}_l .
%\ee

From this we get the power spectrum of the lensing component of the B-modes:
\be
C_{\ell}^{BB,\mathrm{lens}}  = \int \frac{d^2 \bl'}{(2 \pi)^2} W^2 (\bl,\bl')C^{EE}_{l'} C^{\kappa \kappa}_{|\bl-\bl'|}.
\ee

The B-mode power spectrum measured on the sky is composed of a possible primordial component $C_{\ell}^{BB,r}$ together with the lensing $C_{\ell}^{BB,\mathrm{lens}}$ contribution and the instrumental noise $N_{\ell}^{BB}$ (defined in \refeq{noise}):
\be
C_{\ell}^{BB,\mathrm{full}} = C_{\ell}^{BB,r} + C_{\ell}^{BB,\mathrm{lens}} + N_{\ell}^{BB}.
\ee
The lensing component is a significant source of B-modes that, at large scales, correspond to a white noise source of roughly $5 \mu K$-arcmin. This means that it is not only bigger than the allowed inflationary component at scales smaller than several degrees, but it is also comparable with current levels of instrumental noise. 
For this reason, it is critical to characterize and eventually remove it from the data.
To do so, we build a template \refeq{blens} of the lensing B-modes in the observed patch given a measurement of the E-mode field and the lensing potential $\phi$.
While E is measured directly, we can estimate $\phi$ using "tracers" of the dark matter distribution that sources the potential. 

We will now show how the delensing efficiency is related to the fidelity of the lensing tracers and the instrumental noise in the CMB E-modes.
If we have a large scale structure field I($\nver$) that traces the lensing potential we can build a template of the lensing B-modes on the sky by a weighted convolution:
\be
\hat{B}^{\mathrm{lens}}(\bl) = \int \frac{d^2 \bl'}{(2 \pi)^2} W(\bl,\bl') f(\bl,\bl') E^N(\ublu) I(\bl - \bl'),
\ee
where $f(\bl,\bl')$ can be determined by minimizing the difference with the true $B^{\mathrm{lens}}(\bl)$ defined in \refeq{blens}. We introduce the instrumental noise in the CMB E-modes ($E^N$) that will also limit the ability to fully reconstruct the lensing B-modes. 

The residual lensing B-modes due to an imperfect knowledge of the true E-mode and $\phi$ will be
\beqn
B^\mathrm{res}(\bl) &=&  B^\mathrm{lens}(\bl) - \hat{B}^\mathrm{lens}(\bl) =  \int \frac{d^2 \bl'}{(2 \pi)^2} W(\bl,\bl') \times \nonumber \\ &&\left( E(\ublu) \kappa(\bl - \bl' ) -  f(\bl,\bl') E^N(\ublu) I(\bl - \bl' ) \right).
\eeqn

The optimal weights $f(\bl,\bl')$ such that the residual lensing B mode power is minimized are:
\be
\label{eqn:fweight}
f(\bl,\bl') = \left(\frac{C^{EE}_{{l'}}}{C^{EE}_{{l'}}+N^{EE}_{{l'}}}\right)  \frac{C^{\kappa I}_{|\bl-\bl'|}}{C^{II}_{|\bl-\bl'|} }.
\ee
Here $C^{\kappa I}$ and $C^{II}$ are the cross-correlation spectrum of the tracer $I$ with the lensing convergence $\kappa$ and its autospectrum; they are described for each LSS field in \refsec{model}. The power spectrum of the E-modes noise $N^{EE}$ is the same as the B-modes one in \refeq{noise}.

Notice that the first term consists of an inverse variance filter applied to the measured E-mode and the second minimizes the difference between the reconstructed $\phi$ and the CMB lensing potential.

With this choice of $f(\bl,\bl')$ we finally have that the residual power is:
\beqn
\label{eqn:Bres}
C_{{l}}^{BB,\mathrm{res}} &=&   \int \frac{d^2 \bl'}{(2 \pi)^2}  W^2 (\bl,\bl')
 C^{EE}_{l'} C^{\kappa \kappa}_{|\bl-\bl'|}  \\
&\times&  \left[1 - \left(\frac{C^{EE}_{{l'}} }{C^{EE}_{{l'}}+N^{EE}_{{l'}}}\right) \rho^2_{|\bl-\bl'|} \right] \nonumber
\eeqn
with
\be\label{eqn:rho1}
\rho_{\ell}^{2}= \frac{(\cl^{\kappa I})^{2}}{\cl^{\kappa \kappa}\cl^{I I}}.
\ee

\refeq{Bres} directly relate the residual power after delensing with the cross correlation coefficients with CMB lensing of the tracers used.
The bigger the $\rho_{\ell}^{2}$ is for an LSS field, the more it is correlated with the lensing potential acting on the CMB photons. An higher correlation allows for a better reconstruction of $\phi$ and, as a consequence, of $B^{\rm{lens}}$ leading to a smaller residual power $C_{{l}}^{BB,\mathrm{res}}$.
\relook{We conclude this section showing in \reffig{bbplot} the expected residual lensing B-modes power spectrum for some of the tracers used in this work together with the primordial B-modes component and the instrumental noise for current and future experiments.
}



\begin{figure}[htbp]
\begin{center}
\includegraphics[scale=1.2,clip]{../images/BB_res_ell2.pdf}
\caption{Here we illustrate the effect of delensing on the B-mode power spectrum. The red solid line corresponds to the fiducial lensing B-mode component of the signal while the dashed green line corresponds to the inflationary one for the higher amplitude allowed by current experiments.
The residual power left after delensing for some of the LSS tracers described in \refsec{model}. 
The rapid improvement in the level of instrumental noise (dashed curves for SPTPol and CMB S4) will require a high level of delensing to exploit these experiments fully.}
\label{fig:bbplot}
\end{center}
\end{figure}


\section{Multiple tracers of the lensing potential}\label{sec:galcontrib}

In this section we extend the formalism to the case where multiple tracers are used to reconstruct the lensing potential. 
%Furthermore we show that splitting galaxy surveys into multiple tomographic bins can significantly improve their efficiency.

We start by assuming that we have n different tracers of the gravitational potentials $I_{i}$ with $i\in \{1,..,n\}$. We can optimally combine them to estimate $\phi$ or, in other word, maximizing the correlation factor $\rho$ with:
\beqn
I         &=&    \sum_{i}c^{i}I^{i} \nonumber \\
c_{i}     &=& (C_{II}^{-1})_{ij}C^{\kappa I^{j}}
\label{eqn:combined}
\eeqn
where $C_{II}$ is the covariance matrix of the LSS tracers.
The residual B-mode power can be derived from \refeq{Bres} using an ``effective'' correlation $\rho^{2}$ of these combined tracers with gravitational lensing:
\be
\rho^{2}_{\ell} = \sum_{i,j}\frac{C^{\kappa i}_{\ell}~(C^{-1}_{\ell})_{ij}~C^{\kappa j}_{\ell}}{C^{\kappa\kappa}_{\ell}}.
\label{eqn:rho-combined}
\ee

Note that the gain in adding a new tracer is not only proportional to its correlation with the CMB lensing, but it also depends on how much it is correlated with the already used set of tracers.
\relook{
\reffig{kernel} show the different kernels as a function of redshift computed using the models and parameters described in \refsec{model}. Note that the cross correlation of a tracer with the CMB lensing is directly proportional to the overlap of their kernels. 
CIB and 21 cm surveys probes the high redshift structures and, independently from the model assumed, they show a relatively good overlap with the CMB lensing kernel. 
On the other end, galaxy clustering surveys can only reconstruct the low-z portion of the lensing kernel as can be seen from the LSST, DES and DESI curves.
However, given the low noise of this measurement and their small overlap with other probes they can still play an important role in delensing even if their overlap with the CMB lensing potential is not optimal.
} 

% ===================
\begin{figure}[htbp]
\begin{center}
\includegraphics[scale=1.2,clip]{../images/compare_kernel.pdf}
\caption{\textbf{Kernels Comparison}: Comparison of the different kernels as a function of redshift for some of the tracers used in this analysis. The bigger the overlap with the CMB lensing kernel the better the reconstruction of the lensing potential will be leading to a higher delensing efficiency. }
\label{fig:kernel}
\end{center}
\end{figure}
% ===================

\section{Improving efficiency with tomographic binning}\label{sec:tomo-bin}
The delensing efficiency of galaxy surveys can be improved by taking into account redshift information.
When we weight a tracer with $\frac{C^{\kappa I}}{C^{II}}$ in \refeq{fweight} in order to maximize its ability to reconstruct the lensing potential we are only using redshift averaged information about the survey. However, as can be seen in \reffig{kernel}, the kernel overlap of a tracer with the CMB lensing varies as a function of redshifts. For this reason, the optimal approach is to weight galaxies at different redshifts with different weights according to both their cross-correlation with $\kappa$ and their auto-spectrum.
We can see this with a simple example. Let's split a single survey I into two non-overlapping redshift bins $I_{1}$ and ${I_{2}}$ with $I=I_{1}+I_{2}$.
For the full survey the effective cross-correlation is equal to 
\be
\rho_{\rm{full}}^{2}= \frac{(\cl^{\kappa I})^2}{\cl^{\kappa \kappa} \cl^{I I}}=\frac{(\cl^{\kappa I_{1}}+\cl^{\kappa I_{2}})^2}{\cl^{\kappa \kappa} (\cl^{I_{1} I_{1}}+\cl^{I_{2} I_{2}})}
\ee
while for the split survey it will be 
\be
\rho_{\rm{split}}^{2}= \frac{(\cl^{\kappa I_{1}})^2}{\cl^{\kappa \kappa} \cl^{I_{1} I_{1}}}+\frac{(\cl^{\kappa I_{2}})^2}{\cl^{\kappa \kappa} \cl^{I_{2} I_{2}}}.
\ee
Now it can be show that $\rho_{\rm{split}}\ge \rho_{\rm{full}}$ being:
\be
\rho_{\rm{split}}^{2} - \rho_{\rm{full}}^{2}\propto  (\cl^{I_{1} I_{1}}\cl^{\kappa I_{2}}-\cl^{I_{2} I_{2}}\cl^{\kappa I_{1}})^2
\ee
Then $\rho^2$ is always bigger in the tomographic case, and the two are equal only when $\frac{C^{\kappa I_{i}}}{C^{I_{i}I_{i}}}$ is the same for all the redshift bins in which case a single optimal weight is sufficient for the entire survey. 
Of course, binning will improve the efficiency of galaxy surveys, but it is not applicable to tracers with poor redshift information like the CIB or radio continuum surveys. 

In this work, we bin both photometric and spectroscopic galaxy surveys by splitting the window function \refeq{wg} into different slices such as all the bins contain the same number of galaxies. 
For photometric surveys like DES and LSST, we assume a photometric redshift estimation gaussianly distributed around the true value with an rms fluctuation $\sigma(z)$.

In that case the i$^{th}$ slice has a galaxy distribution \cite{hu:2004}:
\begin{eqnarray}
\label{eqn:zbin}
W_{i}(z) \propto b(z) {\frac{dN(z)}{dz}} 
\Bigl[{\,\rm erfc}\left({\Delta(i-1) -z \frac{\sigma(z)}{\sqrt{2}}}\right) 
\\ \nonumber - {\,\rm erfc}
\left({\Delta i  - \frac{z}{\sigma(z)}\sqrt{2}}\right)\Bigr].
\end{eqnarray}

% ===================
\begin{figure}[htbp]
\begin{center}
\includegraphics[scale=1.2,clip]{../images/compare_kernel_tomo.pdf}
\caption{\textbf{Kernels Comparison with tomographic bins}: Comparison of the 10 LSST  tomographic bins together with the full LSST survey and the CMB lensing kernel. 
Compared to a full survey approach, tomographic binning allows to optimally weight different bins according to their cross correlation with the CMB lensing. This leads to a better delensing efficiency. }
\label{fig:kerneltomo}
\end{center}
\end{figure}
% ===================



For photometric surveys, the maximum number of bins is dictated by the fact that the bin width can not be smaller than the photo-z accuracy.
We used 10 and 4 photometric bins for LSST and DES respectively with a common photo-z accuracy $\sigma(z) = 0.05(1+z)$. 
\relook{In spectroscopic surveys, there are no limitations in increasing the number of bins. We split DESI into 4 spectroscopic bins with no overlap among each others.
This number of bins is close to the saturation point where adding more bins does not improve delensing significantly while adding complexity to the analysis.} As an example, we show in \reffig{kerneltomo} the 10 bins and the full LSST redshift distributions together with the CMB lensing kernel. 
\relook{The improvement obtained by tomographic binning is illustrated in \reffig{rho-tomo}. In particular on large scales, binning can increase the value of $\rho$ by almost $30\%$ significantly improving the delensing efficiency of galaxy surveys.}






\begin{figure}[htbp]
\begin{center}
\includegraphics[scale=1.2,clip]{../images/B_res_bin.pdf}
\caption{Tomographic bins improve the cross-correlation of galaxy surveys with CMB lensing. Here we show the cross-correlation coefficient (\refeq{rho1}) as a function of angular scale for full surveys (solid lines) and tomographically binned surveys (dot-dashed line).}
\label{fig:rho-tomo}
\end{center}
\end{figure}


%=================
% FORECAST
%=================



\chapter{Parameter constraints improvement after delensing}
\label{sec:for}

In this section, we forecast the expected delensing efficiency and the relative importance of galaxy tracers for delensing in current and future experiments. We will use the Fisher information matrix to quantify the delensing efficiency as the improvement in the constraint of two inflationary parameters: the tensor to scalar ratio $r$ and the tensor tilt $n_{T}$.
We assume a CMB experimental scenario composed of a high-resolution CMB experiment which defines the internal reconstruction performance, together with a low-noise, low-resolution experiment the B-modes of which \textcolor{red}{(?)} will be delensed and used to constrain the inflationary parameters.
We focus on three distinct scenarios: the current stage, a third generation stage (3G)  and finally the futuristic CMB Stage 4.



%\begin{figure}[htbp]
%\begin{center}
%\includegraphics[scale=1.2,clip]{../images/cmb_internal.pdf}
%\caption{\textbf{CMB internal delensing:} Correlation factor between CMB internal reconstructed potential and the actual lensing potential for different CMB experiments and corresponding noise reconstructions. 
%CMB measurements will rapidly improve providing the most efficient (even if biased) way to reconstruct the structures lensing the CMB.}
%\end{center}
%\label{fig:corr-internal}
%\end{figure}



\section{Fisher Information Matrix}
In the Fisher Information Matrix formalism \cite{10.2307/2342435}, the statistical uncertainty on a cosmological parameter $p$ can be obtained from the inverse of the Fisher matrix $F_{ij}$ as $\sigma(p) =\sqrt{(\mathbf{F})_{pp}^{-1}}$.
We constrain the inflationary parameters $p =\{r,n_{t}\}$ with a CMB B-mode spectrum measurement so the Fisher matrix is:
\beq\label{eqn:fisher}
F_{ij} = \sum^{\ell_{max}}_{\ell=\ell_{min}}\frac{1}{\sigma( C_{\ell}^{BB} )^2}\frac{\partial {C}_{\ell}^{BB}}{\partial p_{i}}\frac{\partial {C}_{\ell}^{BB}}{\partial p_{j}}
\eeq
where we assume a Gaussian covariance:
\beq\label{eqn:cova}
\sigma( C_{\ell}^{BB} )= \sqrt{ \frac{2}{(2 {l}+1) \fsky }} \left( C_{\ell}^{BB,r} + C_{{l}}^{BB,\mathrm{lens}}+N_{{l}}^{BB} \right).
\eeq
The B-modes noise spectrum is given by \cite{knox:1995}:
\beq \label{eqn:noise}
N_{\ell}^{BB} = \left({\Delta_P}/{T_{\mathrm{CMB}}}\right)^2  e^{ {l^2 \theta_\mathrm{FWHM}^2}/({8 \ln 2})}
\eeq
where $\theta_\mathrm{FWHM}$ is the full half width of the telescope beam, and $\Delta_P$ is the instrumental noise of the experiment.

It can be seen in \refeq{fisher} that removing the lensing contribution will improve parameters constraints.
The parameter uncertainties are inversely proportional to the covariance of the measurement. Since the lensing B-modes $C_{{l}}^{BB,\mathrm{lens}}$ are a substantial component of the covariance, removing part of them trough delensing will reduce the parameter statistical error.
After delensing the error $\sigma^{\rm{del}}(p)$ can still be obtained from \refeq{fisher} with $C_{{l}}^{BB,\mathrm{res}}<C_{{l}}^{BB,\mathrm{lens}}$ in the covariance defined in \refeq{cova}.
We define the improvement as the ratio of the constraints before and after delensing: $\alpha_{r} = \sigma^{\rm{del}}(r)/\sigma(r)$ and $\alpha_{n_{t}} = \sigma^{\rm{del}}(n_{t})/\sigma(n_{t})$.

In \refeq{fisher} we are making a few important assumptions. 
Firstly, we are fixing all the cosmological parameters except $\{r,n_{t}\}$. Uncertainties in those will, however, propagate to uncertainties in $\{r,n_{t}\}$. While neglecting this will lead to slightly optimistic constraints, it has no significant impact on the estimate of the improvement due to delensing.

We are also neglecting an important contribution to the measured CMB B-modes: galactic polarized foregrounds. 
The amplitude of these has been constrained in \cite{bicep2/keck-collaboration:2015, planck-collaboration:2015} and it strongly varies in different parts of the sky. Future experiments will use multi-band data to exploit the frequency dependence of these contaminants to remove them from the data. The amount of residual foregrounds depends both on uncertain foreground properties and experimental choices (see a review in \cite{abazajian:2016}).
For this reason, accurately treating foreground requires the use of simulations and the knowledge of several experimental details. 
We decided to focus on an ideal situation assuming no foregrounds or perfect cleaning even if the importance of delensing will be slightly overestimated. 

Furthermore, we are not considering the uncertainties on galaxy survey internal parameters such as biases, source distributions and photometric redshift uncertainties. These uncertaintes can significantly degrade inflationary constraints \cite{sherwin:2015,namikawa:2016a}.
However, as shown in \cite{sherwin:2015,namikawa:2016a}, these can be auto-calibrated i.e. they can be tightly constrained using galaxy survey auto- and cross-correlation spectra. We checked this for a few of the tracer combinations used here and find it particularly true once several tracers are jointly taken into account.
Finally, we have assumed a Gaussian covariance even if the covariance structure has non-Gaussian contributions \cite{motloch:2017,benoit-levy:2012}. This approximation is good enough to show the improvement due to delensing.



%===================
\section{Delensing with current CMB and LSS}

% \the\textwidth


Recently delensing has been performed for the first time on data using both CIB maps \cite{larsen:2016a,manzotti:2017} as well as the internal CMB lensing potential reconstruction \cite{carron:2017} as large scale structure tracers.
Here we discuss the improvement that can be obtained by combining these and other currently available tracers (see \cite{yu:2017} for publicly accessible multi-tracers data-products).

On the CMB side, we will combine a Bicep-Keck like \textit{deep} CMB experiment with an overlapping \textit{higher resolution} experiment. For the deep experiment we assume an instrumental noise equal to \textcolor{red}{$3\mu$K-arcmin}, a beam of \textcolor{red}{30 arcmin} and an angular scale range of $50<\ell<500$.
We assume the CMB lensing reconstruction is  performed by the higher resolution CMB experiment. We explore two possibilities.
First, we test the efficiency with an internal reconstruction of CMB lensing performed by Planck collaboration \cite{planck-collaboration:2016a}. We compute the noise in the CMB lensing map in \refeq{noisekappa} using the noise curves publicly available
\footnote{https://wiki.cosmos.esa.int/planckpla2015/index.php/Specially\_processed\_maps}.
Then we assume a ground-based experiment with noise levels consistent with SPT-Pol: \textcolor{red}{$9.4\mu$K-arcmin} in polarization with a beam equal to \textcolor{red}{1.2 arcmin}. In this case we set the largest achievable scale at $\ell_{min}=300$. However, the results are robust against this choice since the neglected scales contribute only moderately to the reconstruction of the CMB lensing potential.

We combined these CMB experiments with the CIB and current low redshift galaxy surveys like DES and WISE and we compute the improvement in the delensing efficiency.
Following \cite{yu:2017} we cut both the CIB and WISE at $\ell<100$ where they are contaminated by large Galactic dust residuals.
Optical surveys galaxies like DES are less affected by dust and can be used at larger scales.
The achievable correlation is shown in \reffig{corrnow-planck} for Planck and in \reffig{corrnow} for SPT-Pol.
In these figures we include a dashed curve that corresponds to (with arbitrary scale) $<C_{\ell}^{\kappa \kappa} \times \frac{\partial C^{BB}_{\ell'}}{\partial{C^{\kappa \kappa}_{\ell}}}>_{\ell ' <100}$, in order to show which scales in $\kappa$ contribute to the $\ell<100$ B-mode power.

In both cases the CMB lensing reconstruction correlates very well at low $\ell$ and it then falls rapidly at smaller scales because of the raise of the reconstruction noise.
For Planck, the internal reconstruction is at most $70\%$ correlated at very large scales and then it falls rapidly to $40\%$ at $\ell=200$. Its correlation is comparable to the one of LSS at almost all scales.
On the other end, the CMB lensing reconstruction from SPT-Pol will be more than 80\% correlated with the true field at $\ell<300$, and only then the correlation becomes comparable with the current LSS surveys.


\reffig{corrnow} and \reffig{corrnow-planck} show that the DES galaxies are effective tracers of the LSS and can, at least in the near future, be used to improve delensing for CMB experiments that overlap with it. 
For example, DES delensing efficiency is higher than WISE firstly because of the lower level of noise and furthermore because DES galaxies are located at slightly higher redshift thus they better overlap with CMB lensing.


\begin{figure}[htbp]
\begin{center}
\includegraphics[scale=1.2,clip]{../images/actual_scenario_planck.pdf}
\caption{Correlation factor between current galaxy survey and internally reconstructed $\phi$ CMB lensing potential as a function of the angular scale $\ell$. 
The dashed curve corresponds to (with arbitrary scale) $C^{\kappa \kappa} \times \frac{\partial C^{BB}}{\partial{C^{\kappa \kappa}}}$, showing which scales contribute to the $\ell<100$ B-mode power.}
\label{fig:corrnow-planck}
\end{center}
\end{figure}

\begin{figure}[htbp]
\begin{center}
\includegraphics[scale=1.2,clip]{../images/actual_scenario.pdf}
\caption{Correlation factor between current galaxy survey and internally reconstructed $\phi$ CMB lensing potential as a function of the multipole $\ell$. }
\label{fig:corrnow}
\end{center}
\end{figure}

Using these correlation levels we can  compute the residual B-mode power after delensing using \refeq{Bres}, and test the consequent improvement on parameter constraints with \refeq{fisher}. 
For current surveys we only consider $p=r$ since the broad beam of the deep BK-like survey does not provide enough leverage to constrain the shape of the B-mode spectra. Note that we still let $n_{t}$ to vary in our fisher calculation even if it is not very degenerate with $r$.
We first compute the improvement with a fiducial value of $r_{\rm{fid}}=0$.
The results are summarized in \reftab{current} and \reftab{current-spt} for the Planck and SPTPol case respectively.
 
\begin{table}
\centering

\caption{$\alpha(r)$: Improvements on $\sigma(r)$ due to delensing for current generation and Planck lensing reconstruction.
The values in parenthesis in the first column correspond to the fraction of lensing B-mode power removed using each LSS tracer. 
The values in the other columns correspond to ratio of the error before and after delensing.}
  \vspace{0.2cm}
  \begin{tabular}{|c | c | c | c|}
\hline
Surveys & $\alpha(r=0), N^{B}_{\ell}=0$ & $\alpha(r=0)$ & $\alpha(r=0.12)$ \\ \hline \hline
WISE (6\%)& 1.09& 1.05 & 1.01 \\ \hline
DES (14\%) &1.2 & 1.10  & 1.03 \\ \hline
CIB (27\%) &1.45& 1.21  & 1.07\\ \hline
LSS (36\%) &1.7& 1.29  & 1.09\\ \hline
CMB Planck (8\%) &1.06& 1.05  & 1.01 \\ \hline
LSS+CMB (41\%)& 1.77& 1.31& 1.10  \\ \hline
\end{tabular}
\label{tab:current}
\end{table}

We also test a scenario where primordial gravitational waves are present at their highest possible value of $r_{\rm{fid}}=0.12$ \cite{bicep2/keck-collaboration:2015}. As expected the importance of delensing itself is now reduced given that the lensing component constitutes a smaller portion of the total B-modes variance.

\reftab{current-spt} that uses SPTPol instead of Planck for the internal reconstruction shows consistent results.
As expected the improvement of the internal CMB reconstruction reduces the relative importance of LSS. 
This can be seen in \reftab{current-spt}. 
In particular optical surveys will rapidly lose the role of filling in large-scale modes and they will start to supplement information at higher multipoles.
 
 
\begin{table}
\centering
\caption{$\alpha(r)$: Current generation improvements on r due to delensing. Values correspond to the ratio of the error before and after delensing.}
\vspace{0.2cm}
\begin{tabular}{|c | c | c | c|}
\hline
Surveys &$\alpha(r=0), N^{B}_{\ell}=0$& $\alpha(r=0)$ & $\alpha(r=0.12)$ \\ \hline \hline
CMB SPTPol (35\%) && 1.62  &  \\ \hline
LSS+CMB (56\%) && 2.35 &   \\ \hline
\end{tabular}
\label{tab:current-spt}
\end{table}
 

%%%========================
%%%========================





\section{CMB-S3 Era}
CMB polarization measurements are rapidly improving. Indeed the next generation of ground-based telescopes has been already deployed, and data are currently being taken.
We model the CMB S3 stage with a deep experiment with a level of noise of \textcolor{red}{$3\mu$K-arcmin} with a \textcolor{red}{1 arcmin} beam. 
We assume here that a deep CMB experiment and an overlapping high resolution one will be able to combine their measured modes.
For this reason, we will also assume this level of noise for the internal noise reconstruction. With this experimental setup we can push the angular scale range used to calculate the Fisher matrix to $50<\ell<3000$ even if most of the high $\ell$ scale do not contribute to the constraints.
Not only will CMB experiments improve shortly: eventually DESI will start taking data. For this reason, we add DESI to the previously mentioned LSS tracers. 

The correlation factor attainable using generation 3 experiments is shown in \reffig{corrS3}.
An interesting finding is that DESI will be less efficient (removing 10\% of the power) than a DES-like survey (14\%) despite the fact that it can probe slightly higher redshift. The reason is that, for the broad CMB kernel, spectroscopic redshift accuracy is not needed and the lower shot noise in DES play a significant role in the delensing efficiency. Unfortunately, adding DESI will only bring the power removed using LSS fro 36 to 41\%.
Indeed, even in the near future, CIB will still play the dominant role among LSS.
Contrary to current experiments, th CMB internal reconstruction will dominate the correlation with the lensing potential up to $\ell\simeq 550$.

%This generation of experiments will start exploring the small scales B-modes thus constraining, even if loosely, the value of $n_{t}$.

\begin{figure}[htbp]
\begin{center}
\includegraphics[scale=1.2,clip]{../images/S3_scenario.pdf}
\caption{Correlation factor. Same as \reffig{corrnow} but for stage 3 experiments. }
\label{fig:corrS3}
\end{center}
\end{figure}

 
The improvement in the constraint on $r$ is summarized in \reftab{S3}. 
Given the improvement in the noise of the high resolution experiment, CMB alone will be able to improve constraint on $r$ by a factor of 2 through internal delensing in the ideal case of no instrumental noise.
Despite this, adding galaxy survey will still lead to a decent improvement with a fairly low effort.
Indeed galaxy surveys will still be able to remove an additional $20\%$ of power from $C_{\ell}^{BB,\mathrm{lens}}$ and to improve our constraint on the null hypothesis ($r=0$) by $15\%$. Note that apart from the addition of DESI, the tracers used here are already available for the current generation. The ongoing effort can then lead to a significant improvement even for the next generation of experiments. 
Finally, the combination of CMB and galaxy surveys will be able to improve the constrain on the null hypothesis of no primordial waves by a factor 1.8 or to improve the constraint of a possible detection (r=0.12) by 30\%.
As expected the improvement on $n_{t}$ is similar to the one in $r$ since they are both proportional to the variance of the measured B-modes. We tested this in the scenario $r=0.12$ where we additionally impose the consistency relation $n_{t}^{\rm{fid}}=-r^{\rm{fid}/8}$. 
It is important to note that, independently from the level of delensing, for CMB 3G experiments this statistical error will still be several times bigger than the fiducial value.  



\begin{table}
\centering

\caption{$\alpha$: S3 improvements. Note that the absolute error value of $\sigma(r)$ after delensing for $f_{sky} = $ is $\sigma(r) =$  }
  \begin{tabular}{|c | c | c | c|}
\hline
Surveys & $\alpha(r=0), N^{B}_{\ell}=0$ & $\alpha(r=0)$ & $\alpha(r=0.12)$ \\ \hline \hline
DESI (10\%) &1.12 & 1.08  & 1.03 \\ \hline
LSS (40\%) &1.73& 1.42  & 1.2\\ \hline
CMB 3G (56\%) &2& 1.57  & 1.21 \\ \hline
LSS+CMB (69\%)& 2.8 & 1.8& 1.31  \\ \hline
\end{tabular}
\label{tab:S3}
\end{table}


\section{CMB-S4 Era}

An ambitious program for a generation 4 ground CMB experiment is currently under planning \cite{abazajian:2016}. 
Moreover, satellite and balloon experiments have been proposed and have the potential to extend the accessible B-mode measurement to the largest scales.
Here we follow \cite{abazajian:2016} and assume a CMB-S4 ground experiment with  with a level of noise of \textcolor{red}{$1\mu$K-arcmin} with a \textcolor{red}{1 arcmin} beam.
By the mid 2020s several next generation of LSS survey will be online. Here, as an example for future optical galaxy surveys we add LSST \cite{lsst-science-collaboration:2009} to the CMB lensing tracers. We test that other experiments like Euclid \cite{laureijs:2011} and WFIRST \cite{spergel:2013} will have similar delensing performance. Even if they will observe similar part of the sky, because of the different  strength and weakness of this experiments  
it will still important to combined them efficiently.
We also consider a radio continuum survey modeled following the SKA planned experiment. Radio continuum observation of the 21 cm line is in its early stage and several experimental and analysis challanges 
need to be overcome. However this technique has the potential to map LSS with relatively low noise up to redshift $z=6$.

The correlation factor with CMB lensing of stage 4 experiments is shown in \reffig{corrs4}.

\begin{figure}[htbp]
\begin{center}
\includegraphics[scale=1.2,clip]{../images/S4_scenario.pdf}
\caption{Correlation factor. Same as \reffig{corrnow} but for stage 4 experiments. }
\label{fig:corrs4}
\end{center}
\end{figure}

%\begin{figure}[htbp]
%\begin{center}
%\includegraphics[scale=1.2,clip]{../images/S4_scenario_no_ska.pdf}
%\caption{Correlation factor. Same as \reffig{corrnow} but for stage 4 experiments without SKA. }
%\label{fig:corrs4}
%\end{center}
%\end{figure}


A tomographically binned LSST-like experiment will be a very efficient CMB lensing tracer. Indeed it will be more than 70\% correlated with CMB lensing for all the scales $\ell<600$ a performance similar to a Stage 3 CMB internal reconstruction.
SKA very new but potentially amazing
With SKA and LSST, for the first we will have LSS tracers way more efficient than the CIB.
Even if galaxy survey will improve dramatically the CMB internal reconstruction will still be the main source of delensing. This, having a perfect kernel overlap with the true lensing potential we will benefit for the very low level of noise of CMB S4 experiments.
CMB S4 delensing will probably be limited by small secondary effect like foregrounds contamination, missing modes due to filtering etc.

\reftab{S4} shows the improvement on the inflationary constraint. 


%=======================================

\begin{table}
\centering

\caption{$\alpha$: Gen 4 experiments}
  \begin{tabular}{|c | c | c | c|}
\hline
Surveys & $\alpha(r=0), N^{B}_{\ell}=0$ & $\alpha(r=0)$ & $\alpha(r=0.12)$ \\ \hline \hline
LSST (10\%) &1.12 & 1.08  & 1.03 \\ \hline
SKA (40\%) &1.73& 1.42  & 1.2\\ \hline
CMB S4 (56\%) &2& 1.57  & 1.21 \\ \hline
LSS+CMB (69\%)& 2.8 & 1.8& 1.31  \\ \hline
\end{tabular}
\label{tab:S4}
\end{table}


%=======================================

\begin{table}
\centering

\caption{$\alpha$: Gen 4 experiments}
  \begin{tabular}{ | c | c  |  c |}
\hline
Surveys & $\alpha(r)$ & $\alpha(n_{t})$\\ \hline \hline
Euclid & 1.53 & \\ \hline
LSST & 1.65  &\\ \hline
SKA & 4.35  &\\ \hline
CMB S4 & 3.7  &\\ \hline
LSS+CMB & 6.3  &\\ \hline
\end{tabular}
\label{tab:S4}
\end{table}


%\section{Bias uncertainties degradation}
%The uncertainties in the theoretical assumptions used to model the galaxies can cause a degradation of the improvement of the inflationary constraint of delensed spectra.
%In this section, we quantify this effect.
%We will now marginalize over unknown galaxies parameters, but we will use a full dataset of CMB and galaxies data. The idea is that as shown in the low level of noise in galaxy surveys might allow us to calibrate them internally.
%We will use a Fisher approach the Fisher matrix is:
%\beqn
%F_{pq} &=& \sum_{l_{a} =l^{BB}_{\mathrm{min}}}^{l^{BB}_{\mathrm{max}}}  \sum_{l_{b} =l^{BB}_{\mathrm{min}}}^{l^{BB}_{\mathrm{max}}}   \frac{\partial {C}_{l_a}^{BB,\mathrm{del}}}{\partial \theta_p } \left[ \mathrm{Cov}^{BB,BB}\right]_{l_a, l_b} ^{-1} \frac{\partial {C}_{l_b}^{BB,\mathrm{del}}}{\partial \theta_q }\nonumber \\&+&  \sum_j \frac{\frac{\partial {C}_j^{\kappa I}}{\partial \theta_p } \frac{\partial {C}_j^{\kappa I}}{\partial \theta_q }}{(\Delta C_j^{\kappa I})^2}
%+\sum_j \frac{\frac{\partial {C}_j^{I I}}{\partial \theta_p } \frac{\partial {C}_j^{I I}}{\partial \theta_q }}{(\Delta C_j^{I  I})^2}
%\eeqn
%
%\beq
%\alpha_{\mathrm{marginalized}} = \sigma_0(r) / \sigma_{\mathrm{marginalized/delensed}}(r)
%\eeq
%
%
%where the parameter array contains both the tensor to scalar rate $\theta=r$, and the galaxy surveys parameters like the bias $b_i$ or $p_i$ \footnote{In our analysis, since the fiducial value of $r$ is zero,
%the derivative of $C_\ell^{\rm BB,res}$ is non-zero if $\theta_i=r$.}.
%
%We compute the derivatives of the power spectra as described


%\begin{table}
%\caption{$\alpha$: improvement on r constraint}
%  \begin{tabular}{ | c | c  | }
%\hline
%Surveys & $\alpha$\\ \hline \hline
%des & 1.34  \\ \hline
%cib & 1.71  \\ \hline
%cmb current & 1.79  \\ \hline
%gals current & 2.03  \\ \hline
%cmb S3 & 2.15  \\ \hline
%gals S3 & 2.16  \\ \hline
%gals S4 & 2.16  \\ \hline
%comb current & 2.66  \\ \hline
%comb S3 & 3.14  \\ \hline
%cmb S4 & 4.36  \\ \hline
%comb S4 & 5.27  \\ \hline
%  \end{tabular}
%\end{table}


\chapter{Conclusions}
\label{sec:concl}

The ability to separate the lensing component of the CMB B-modes from a possible primordial inflationary signal ("delensing") is necessary to test the inflationary paradigm using the next generation of CMB polarization experiments. 
To do so, we need to accurately reconstruct the large scale structures that lens the CMB in order to build a template of the lensing B-modes.
In this paper, we studied the potential impact of large-scale structure galaxy surveys in this important endeavor.
%For current experiment, the cosmic infrared background had already proven to be very efficient in delensing \cite{larsen:2016,manzotti:2017}. 
For ongoing experiments, we find that an optical survey like DES alone is able to remove $14\%$ of the lensing component and together with WISE($8\%$) and the CIB ($27\%$) will allow to remove $36\%$ of the contaminant signal using only LSS survey.
Depending on the CMB noise and the amount of galactic foreground cleaning these can correspond to an improvement of 30\% in the constrain on the tensor to scalar ratio $r$.

In the future, the improved level of noise of CMB experiments will allow to internally reconstruct the structures lensing the CMB. 
The fraction of removed lensing B-modes will rapidly improve from the current level of Planck ($8\%$) and SPTPol ($35\%$) to 3G (56\%) and CMB S4 (76\%) level. 
Already in the 3G era, the CMB internal reconstruction will be the main source of delensing. However it will still be less efficient than galaxies at small scales$\ell>500$. For this reason, combining galaxy survey with the CMB will push the fraction of removed power from $57\%$ to $68\%$.

Furthermore internal delensing will require a careful study of possible biases and systematics effects due to the use of the same source (the CMB) we are delensing to reconstruct the lensing effect itself \cite{carron:2017,sehgal:2016,namikawa:2017}.
For this reason, efficient galaxies tracers are not only useful in the short term but will also play a role in cross-checking these internal biases. 
For a future galaxy survey like LSST we find that using tomographic binning can improve its correlation with lensing by $30\%$.

Giving the high level of correlation of future galaxy surveys and CMB lensing it is worth looking for possible application beside the detection of inflationary B-modes.
For example LSS delensing allow to remove just the low-z lensing component from the CMB lensing. Removing this component can avoid the necessity of modelling nonlinearities in the study of CMB lensing power spectrum.

% ===================
%\onecolumngrid
\begin{figure}[htbp]
\begin{center}
\includegraphics[scale=1.2,clip]{../images/errors_summary.pdf}
\caption{\relook{to be filled with final results}}
\label{fig}
\end{center}
\end{figure}
\twocolumngrid



\makebibliography

\end{document}