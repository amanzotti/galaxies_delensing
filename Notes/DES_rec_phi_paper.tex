\input{/Users/alessandromanzotti/Work/preamble_onecol_notitle.tex}
\usepackage{hyperref}
\usepackage{amsmath}
\def\ublu{\bl'}
\newcommand{\bx}{{\bf x}}
\newcommand{\bk}{{\bf k}}
\newcommand{\bq}{{\bf q}}
\newcommand{\bP}{{\bf\Psi}}
\newcommand{\bs}{{\bf s}}
\newcommand{\cs}{{\cal S}}
\newcommand{\by}{{\bf y}}
\newcommand{\deltar}{\delta_{\rm recon}}%\def\lcdm{$\Lambda$CDM}
\def\arcsec{$^{\prime\prime}$}
\def\nl{N_\ell}
\def\sl{S_\ell}
\newcommand{\conv}[2]{\left(\frac{#1}{#2}\right)}
\def\arctanh{\mathop{\rm arctanh}\nolimits}
%\renewcommand{\eqref}[1] {equation $($\ref{#1}$)$}
\newcommand{\comment}[1]{{}}
\def\beq{\begin{equation}}
\def\eeq{\end{equation}}
\def\beqn{\begin{eqnarray}}
\def\eeqn{\end{eqnarray}}
\def\a{\alpha}
\def\cl{C_{l}}
\def\h{\mathrm{h}}
\def\d{\rmn{d}}
\def\pa{\partial}
\def\deldel#1#2{\frac{\pa{#1}}{\pa{#2}}}
\def\ba{\bm{\alpha}}
\def\fracj#1#2{{\textstyle{#1\over#2}}}
\def\bxi{\bm{\xi}}
\def\half{\frac{1}{2}}
\def\ti{\widetilde}
\def\O{\Omega}
\def\OL{\Omega_\Lambda}
\def\Om{\ensuremath{\Omega_{\mathrm{m}}}}
\def\Ob{\ensuremath{\Omega_{\mathrm{b}}}}
\def\Oc{\ensuremath{\Omega_{\mathrm{CDM}}}}
\def\msol{\ensuremath{M_\odot}}
\def\l{\left}
\def\r{\right}
\def\o{\omega}
\def\gcm{\textrm{g cm$^{-3}$}}
\def\2gcm{\textrm{g cm$^{-2}$}}
\def\Scr{\Sigma_{\mathrm{crit}}}
\def\rcr{\rho_{\mathrm{crit}}}
\def\phidot{\ensuremath{\dot\phi_{\bl,\bl'}}}
\def\ddelta{\ensuremath{\dot\delta}}
\def\modu#1{\l |{#1}\r |}
\def\av#1{\l \langle{#1}\r \rangle}
\def\hmpc{\:{h}^{-1}\mathrm{Mpc}}
\def\th{\Theta}
\def\tth{\tilde\Theta}
\def\sg{\sigma}
\def\Sig{\Sigma}
\def\cf{{\cal F}}
\def\k{\kappa}
\def\P{{P}}
\def\pnl{{P}_{{\!\textrm{\tiny NL}}}}
\def\dnl{\Delta_{\text{\scriptsize NL}}}
\def\kmin{\k_{\mathrm{min}}}
\def\kmax{\k_{\mathrm{max}}}
\def\hires{the \emph{highRes} experiment}
\def\lores{the \emph{lowRes} experiment}
\def\ktot{\k_{\mathrm{tot}}}
\def\hunit{\ensuremath{\mathrm{km}{\mathrm{s}^{-1}} \mathrm{Mpc}^{-1}}}
\def\H0{\ensuremath{\mathrm{H}_0}}
\def\nn{\nonumber}
\def\lin{\mathrm{lin}}
\def\ISW{\mathrm{ISW}}
\def\bl{\bmm{l}}
\def\bL{\bmm{L}}
\def\fsky{f_{\mathrm{sky}}}
\newcommand{\E}[1]{\times 10^{#1}}
\newcommand{\bmm}[1]{{\mathbf{#1}}}
\newcommand{\new}[1]{{\color{blue} #1}}
\newcommand{\bsection}[1]{\section{\uppercase{#1}}}


\begin{document}

\title{D.E.S reconstructed lensing potential template and application to delensing and cross correlation }         % Enter your title between curly braces
\author{A. Manzotti}        % Enter your name between curly braces
\date{\today}

        % Enter your date or \today between curly braces
\begin{abstract}
Goal: test if, at least on a small patch of the sky $~500$ deg$^{2}$ LSS survey like D.E.S or DESI can improve the reconstruction of the lensing potential that lenses the CMB photons.
This would be crucial to build a template of the B-mode signal coming from the lensing of the primordial E-mode. This will be at multipoles higher than 200 the main contaminants of the primordial B-mode.
\end{abstract}

\vspace{1cm}

\maketitle


\section{Theory} \label{sec:th}



%=======================================
\begin{figure}[htbp]
\begin{center}
\includegraphics[scale=1.]{../images/spectra.pdf}
\caption{Correlation factor between galaxies survey and CMB lensing potential.}
\label{fig:corr}
\end{center}
\end{figure}
%=======================================


%=======================================
\begin{figure}[htbp]
\begin{center}
\includegraphics[scale=1.]{../images/clbb_res.pdf}
\caption{Residual lensing B modes power spectrum using different large scale structure.}
\label{fig:bres}
\end{center}
\end{figure}
%=======================================


\begin{figure}[htbp]
\begin{center}
\includegraphics[scale=1.]{../images/compare_kernel.pdf}
\caption{Comparison of the different kernels used in this analysis. This allow to understand  
Redshift distribution of D.E.S galaxies (I suspect this is the benchmark, anyway taken from Giannantionio et al.). }
\label{fig:corr}
\end{center}
\end{figure}
As for the temperature, the intensity map of photons on the sky, also the Q and U mode decomponsition of their polarization is modified by lensing as: 
\be
Q(\hat{\mathbf{n}}) = Q_{\mathrm{unlensed}}(\hat{\mathbf{n}}+\mathbf{d});~~
U(\hat{\mathbf{n}}) = U_{\mathrm{unlensed}}(\hat{\mathbf{n}}+\mathbf{d})
\ee
where $\mathbf{d}$ is the deflection angle directly related to the lensing potential $\phi$.

As a first approximation, the B mode resulting from the lensing of primordial E mode by a convergence field $\kappa$ is:
\be \label{eq}
B^{\mathrm{lens}}(\bl) =  \int \frac{d^2 \bl'}{(2 \pi)^2} W(\bl,\bl') E(\ublu) \kappa(\bl - \bl')
\ee
where
\be
W(\bl,\bl') = \frac{2 \bl' \cdot (\bl-\bl')}{|\bl-\bl'|^2} \sin(2\varphi_{\bl,\bl'}),
\ee

As usual we define the power spectrum as:
\be
\langle B^{\mathrm{lens}}(\bl) ~ B^{\mathrm{lens}^*}(\tilde{\bl}) \rangle \equiv (2\pi)^2 \delta^D(\bl - \tilde{\bl}) C^{BB,\mathrm{lens}}_l 
\ee

From this we get that the power spectrum:
\be
C_l^{BB,\mathrm{lens}}  = \int \frac{d^2 \bl'}{(2 \pi)^2} W^2 (\bl,\bl')C^{EE}_{l'} C^{\kappa \kappa}_{|\bl-\bl'|}.
\ee

Now the full B-mode power spectrum measured on the sky is
\be
C_l^{BB,\mathrm{full}} = C_l^{BB,r} + C_l^{BB,\mathrm{lens}} + N_l^{BB}.
\ee

Now, if we have an LSS measurements I that traces the lensing potential responsible for the lensing of the CMB we can build a template of the true lensing B mode on the sky with a weighted convolution:
\be
\hat{B}^{\mathrm{lens}}(\bl) = \int \frac{d^2 \bl'}{(2 \pi)^2} W(\bl,\bl') f(\bl,\bl') E^N(\ublu) I(\bl - \bl')
\ee
where $f(\bl,\bl')$ is a weight that must be determined.

The residual lensing b mode will be 
\beqn
B^\mathrm{res}(\bl) &=&  B^\mathrm{lens}(\bl) - \hat{B}^\mathrm{lens}(\bl) =  \int \frac{d^2 \bl'}{(2 \pi)^2} W(\bl,\bl') \times \nonumber \\ &&\left( E(\ublu) \kappa(\bl - \bl' ) -  f(\bl,\bl') E^N(\ublu) I(\bl - \bl' ) \right) 
\eeqn
and its power spectrum
\beqn
C_{{l}}^{BB,\mathrm{res}} &=&   \int \frac{d^2 \bl'}{(2 \pi)^2} W^2 (\bl,\bl') 
 [ C^{EE}_{l'} C^{\kappa \kappa}_{|\bl-\bl'|} \nonumber \\ && -(f(\bl,\bl')+f^*(\bl,\bl')) C^{EE}_{l'} C^{\kappa I}_{|\bl-\bl'|} \\ \nonumber && + f^*(\bl,\bl') f(\bl,\bl') (C^{EE}_{{l'}}+ N^{EE}_{{l'}}) C^{II}_{|\bl-\bl'|} ] 
 \eeqn
 
 
We can now easily choose $f(\bl,\bl')$ so that the residual lensing B mode power is minimized. We find:
\be
f(\bl,\bl') = \left(\frac{C^{EE}_{{l'}}}{C^{EE}_{{l'}}+N^{EE}_{{l'}}}\right)  \frac{C^{\kappa I}_{|\bl-\bl'|}}{C^{II}_{|\bl-\bl'|} } 
\ee
Notice that the first term consists in the usual inverse variance filter applied to the measured E-mode and the second minimize the difference between the reconstructed $\phi$ and the CMB lensing potential.

We finally have that the residual power is:
\beqn
C_{{l}}^{BB,\mathrm{res}} &=&   \int \frac{d^2 \bl'}{(2 \pi)^2}  W^2 (\bl,\bl') 
 C^{EE}_{l'} C^{\kappa \kappa}_{|\bl-\bl'|}  \\
&\times&  \left[1 - \left(\frac{C^{EE}_{{l'}} }{C^{EE}_{{l'}}+N^{EE}_{{l'}}}\right) \rho^2_{|\bl-\bl'|} \right] \nonumber
\eeqn
with
\be
\rho_l= \frac{\cl^{\kappa I}}{\sqrt{\cl^{\kappa \kappa} \cl^{I I}}}.
\ee

The bigger $\rho_l$ is for a LSS field the more it is correlated with the lensing potential acting on the CMB photons. An higher correlation allows for a better reconstruction of the $\phi^{CMB}$ and, as a consequence, of $B^{lens}$.

\section{Data} \label{sec:data}

D.E.S, Planck CIB, DESI?
\cite{hall:2010,sherwin:2015,ade:2014,hanson:2013,szapudi:2001,0004-637X-567-1-2,planck-collaboration:2014,planck-collaboration:2014a,planck-collaboration:2011,plank-collaboration:2014,boulanger:1996,lewis:2006}

\section{Forecast} \label{sec:for}

\bibliographystyle{abbrv}
\bibliography{DES_rec_phi_paper}

\end{document}
