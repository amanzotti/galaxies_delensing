\documentclass[reprint,superscriptaddress,amsmath,amssymb,floatfix,aps,prd,nofootinbib]{revtex4-1}
\usepackage{bm}
\usepackage{epsfig}
\usepackage{natbib}
\usepackage{graphicx,epsfig}
\usepackage{hyperref}
\usepackage{ifthen}
\usepackage{xstring}
% \usepackage{graphicx}\usepackage[top=2.5 cm, bottom=2.5 cm, left=1.05in, right=1.05in]{geometry}
\usepackage[applemac]{inputenc}
\usepackage{amsmath,amssymb}
\usepackage{color}
\usepackage{epstopdf}
%\usepackage[caption=false]{subcaption}
%\epstopdfDeclareGraphicsRule{.gif}{png}{.png}{convert gif:#1 png:\OutputFile}
% \AppendGraphicsExtensions{.gif}
% notes to make comment is non-zero if
\newcommand{\al}[1]{\begin{align} #1 \end{align}}

\newcommand{\note}[1]{\textsc{#1}}
	\newcommand{\bR}[1]{{\bm {\mR{#1}}} }

\def\l{\ell}
% For various journals
%\newcommand{\apj}{ApJ}
\newcommand{\physrep}{Physics Reports}
\newcommand{\jcap}{JCAP}
\newcommand{\apjl}{ApJL}
\def\araa{\ref@jnl{ARA\&A}}
\newcommand{\apjs}{ApJS}
\newcommand{\aap}{A\&A}
\newcommand{\mnras}{MNRAS}
\newcommand{\physrev}{Phys. Rev.}
\newcommand{\physrevlett}{Phys. Rev. Lett.}
\newcommand*\aj{AJ}
%\newcommand*\prd{Phys.~Rev.~D}




\newcommand{\nver}{\hat{\mathbf{n}}}
\newcommand{\cov}{\text{Cov}}
\newcommand{\Nsim}{N_{\text{sim}}}
\newcommand{\new}[1]{{\textcolor{blue}{\bf #1}}}
%--------- NEW COMMAND STUFF
	\newcommand{\PD}[2]{\dfrac{\partial #1}{\partial #2}}

\def\simlt{\lesssim}
\def\simgt{\gtrsim}
\newcommand{\Omegamzero}{\Omega_{{\rm m,0}}}
\newcommand{\alm}{a_{lm}}
	% * subscript
	\def\rom#1{%
		_{\mathrm{#1}}%
	}%
%\newcommand{\l}{\ell}}

\def\be{\begin{equation}}
\def\ee{\end{equation}}
\def\ben{\begin{equation*}}
\def\een{\end{equation*}}

\def\ba{\begin{eqnarray}}
\def\ea{\end{eqnarray}}
\def\ban{\begin{eqnarray*}}
\def\ean{\end{eqnarray*}}

\newcommand{\refsec}[1]{section~\ref{sec:#1}}
\newcommand{\reftab}[1]{Tab.~\ref{tab:#1}}
\newcommand{\refeq}[1]{Eq.~(\ref{eqn:#1})}
\newcommand{\refssec}[1]{section~\ref{subsec:#1}}
\newcommand{\reffig}[1]{Fig.~\ref{fig:#1}}
\newcommand{\refFig}[1]{Fig.~\ref{fig:#1}}
	\newcommand{\mR}[1]{\mathrm{#1}}   % mathrm



\definecolor{darkgreen}{cmyk}{0.85,0.2,1.00,0.2}
\definecolor{purple}{cmyk}{0.5,1.0,0,0}






\usepackage{hyperref}
\usepackage{amsmath}
\usepackage{natbib}
\newcommand{\bx}{{\bf x}}
\newcommand{\bk}{{\bf k}}
\newcommand{\bq}{{\bf q}}
\newcommand{\bP}{{\bf\Psi}}
\newcommand{\bs}{{\bf s}}
\newcommand{\cs}{{\cal S}}
\newcommand{\by}{{\bf y}}
\newcommand{\deltar}{\delta_{\rm recon}}%\def\lcdm{$\Lambda$CDM}
\def\arcsec{$^{\prime\prime}$}
\def\nl{N_\ell}
\def\sl{S_\ell}
\def\ublu{\bl'}
\newcommand{\conv}[2]{\left(\frac{#1}{#2}\right)}
\def\arctanh{\mathop{\rm arctanh}\nolimits}
%\renewcommand{\eqref}[1] {equation $($\ref{#1}$)$}
\newcommand{\comment}[1]{{}}
\def\beq{\begin{equation}}
\def\eeq{\end{equation}}
\def\beqn{\begin{eqnarray}}
\def\eeqn{\end{eqnarray}}
\def\a{\alpha}
\def\cl{C_{l}}
\def\h{\mathrm{h}}
\def\d{\rmn{d}}
\def\pa{\partial}
\def\deldel#1#2{\frac{\pa{#1}}{\pa{#2}}}
\def\ba{\bm{\alpha}}
\def\fracj#1#2{{\textstyle{#1\over#2}}}
\def\bxi{\bm{\xi}}
\def\half{\frac{1}{2}}
\def\ti{\widetilde}
\def\O{\Omega}
\def\OL{\Omega_\Lambda}
\def\Om{\ensuremath{\Omega_{\mathrm{m}}}}
\def\Ob{\ensuremath{\Omega_{\mathrm{b}}}}
\def\Oc{\ensuremath{\Omega_{\mathrm{CDM}}}}
\def\msol{\ensuremath{M_\odot}}
\def\l{\left}
\def\r{\right}
\def\o{\omega}
\def\gcm{\textrm{g cm$^{-3}$}}
\def\2gcm{\textrm{g cm$^{-2}$}}
\def\Scr{\Sigma_{\mathrm{crit}}}
\def\rcr{\rho_{\mathrm{crit}}}
\def\phidot{\ensuremath{\dot\phi_{\bl,\bl'}}}
\def\ddelta{\ensuremath{\dot\delta}}
\def\modu#1{\l |{#1}\r |}
\def\av#1{\l \langle{#1}\r \rangle}
\def\hmpc{\:{h}^{-1}\mathrm{Mpc}}
\def\th{\Theta}
\def\tth{\tilde\Theta}
\def\sg{\sigma}
\def\Sig{\Sigma}
\def\cf{{\cal F}}
\def\k{\kappa}
\def\P{{P}}
\def\pnl{{P}_{{\!\textrm{\tiny NL}}}}
\def\dnl{\Delta_{\text{\scriptsize NL}}}
\def\kmin{\k_{\mathrm{min}}}
\def\kmax{\k_{\mathrm{max}}}
\def\hires{the \emph{highRes} experiment}
\def\lores{the \emph{lowRes} experiment}
\def\ktot{\k_{\mathrm{tot}}}
\def\hunit{\ensuremath{\mathrm{km}{\mathrm{s}^{-1}} \mathrm{Mpc}^{-1}}}
\def\H0{\ensuremath{\mathrm{H}_0}}
\def\nn{\nonumber}
\def\lin{\mathrm{lin}}
\def\ISW{\mathrm{ISW}}
\def\bl{\bmm{l}}
\def\bL{\bmm{L}}
\def\fsky{f_{\mathrm{sky}}}
\newcommand{\E}[1]{\times 10^{#1}}
\newcommand{\bmm}[1]{{\mathbf{#1}}}
\newcommand{\bsection}[1]{\section{\uppercase{#1}}}


\begin{document}

\title{Future Cosmic Microwave Background delensing with galaxies surveys.}         % Enter your title between curly braces
\author{A. Manzotti}        % Enter your name between curly braces
\email{manzotti.alessandro@gmail.com}
\affiliation{Kavli Institute for Cosmological Physics, University of Chicago}
\affiliation{Department of Astronomy and Astrophysics, University of Chicago, 5640 South Ellis Avenue, Chicago, IL 60637, USA}
\date{\today}

        % Enter your date or \today between curly braces
\begin{abstract}
%\begin{itemize}
%\item Worry : they might ask for internal bias\
%item Possible TODO : temperature $N_{eff}$
%
%\end{itemize}
%

The measurement of the cosmic microwave background (CMB) polarization is a promising experimental dataset to test the inflationary paradigm and to probe the physics of the early universe. 
A particular component, the so-called B-modes is indeed a direct signature of the presence of gravitational waves in the early universe. However improving the level of noise is not enough. 
This is even truer if the aim is to not only detect the amplitude of gravitational waves but also the shape of their spectrum to test for example inflation consistency relations.
Removing the lensing component from the measurement of CMB B-modes will be important to constrain the amplitude of the primordial gravitational wave contribution to the signal.
Here we discuss the role of current and future large scale structure surveys in improving the reconstruction of the lensing potential that lenses the CMB photons and how this reflects into an improved delensing efficiency.
We quantify this by the improvement due to delensing on the constraints on the inflationary tensor perturbations amplitude and shape ($r$ and $n_{t}$)
We find that..
\end{abstract}

\vspace{1cm}

\maketitle

\section{Introduction}
\label{sec:intro}
In the standard cosmological paradigm, the universe underwent a period of near-exponential expansion in its early phase;
this period is called ``cosmic inflation.''

Inflation generically predicts a stochastic background of gravitational waves \cite[see e.g.][for a review]{kamionkowski15}.
These primordial gravitational waves in the early universe imprint a unique signature on the polarized anisotropies of the CMB.
CMB polarization is usually decomposed fields into even-parity (divergence) and odd-parity (curl) components, referred to as ``E'' and ``B'' modes.
In the standard scenario, the PGW background is the only source of B-mode polarization at the epoch of recombination.

The amplitude of this primordial B-mode component is parametrized by the ratio of the amplitudes of the primordial tensor and scalar perturbations, $r$. The shape of this component, i.e. the amplitude as a function of the angular scales is encoded in the tensor tilt $n_{t}$.
Most of the inflationary models directly relate the value of r to the energy scale of inflation.
CMB B modes provide the cleanest known observational window into the primordial gravitational waves background.
Measuring or constraint the value of $r$ and $n_t$ will put tight constraint on the allowed models of inflation.

Indeed improving the CMB B-modes measurement is a major objective of current and future CMB experiments.
Even if inflationary B-modes have not been detected yet, a possibly natural value $r \gtrsim 10^{-3}$, which should be reachable given the near future level of noise expected in CMB experiments \citep{kamionkowski15}.

However reducing the level of noise will not be enough to reach this goal.
Unfortunately the observed B modes are not solely due primordial gravitational waves but also by late time effects. In this work we will focus on gravitational interactions with LSS.
Lensing shears the CMB polarization pattern, producing ``lensing B modes'' from CMB E modes \citep{zaldarriaga98}.
This expected component has now been measured both in cross-correlation with LSS \citep{hanson13,polarbear2014c,vanengelen14b,planck2015XV} and directly from CMB data alone \citep{polarbear2014b,bicep2a,keisler15}.

This component acts as a source of confusion  for searches of the PGW background.
Indeed, current experiment are already limited by the contamination from lensing B modes since the instrument noise is below the lensing B mode $rms$ fluctuations \citep{bk14}.
The optimal way to approach this source of noise is to build the specific realization of lensing B modes on the sky can be and used it to clean the data in a process called ``delensing''. 

Delensing has been studied for many years \cite{knox2002, kesden2002, seljak2003, simard:2015,sherwin15,smith:2012}.
Furthermore recently it has been performed on CMB temperature data using the cosmic infrared background in \cite{larsen:2016}, on CMB temperature and polarization data using internal CMB reconstruction \cite{ carron17}. Finally the highest B-mode delensing efficiency has been achieved with SPT and Herschel data in  where $~28\%$ of the lensing power has been removed.
\vspace{0.5cm}

In this paper we

\vspace{0.5cm}

This paper is organized as follow: in \refsec{th} we describe the lensing B-mode components and the residual power after delensing with tracers of the lensing potential. We describe the LSS tracers used in this analysis in \refsec{model}. \refsec{for} is the main results of this work: the improvement to inflationary parameters due to delensing with CMB and LSS. We conclude in \refsec{concl}.




\section{Gravitational lensing B-mode and delensing}
\label{sec:th}

As for the CMB photons temperature, the intensity map of photons on the sky, also the Q and U mode decomposition of their polarization is remapped by lensing as:
\be
Q(\hat{\mathbf{n}}) = Q_{\mathrm{unlensed}}(\hat{\mathbf{n}}+\nabla\phi);~~
U(\hat{\mathbf{n}}) = U_{\mathrm{unlensed}}(\hat{\mathbf{n}}+\nabla\phi)
\ee
where the deflection angle is the gradient of lensing potential potential integrated along the line of sight $\nabla\phi$.
The CMB polarization is usually decomposed into odd-parity Fourier modes E and B. 
As shown in \cite{} this for symmetry reason tensor perturbations are the main source of the B-modes configuration. For this reason, B-modes are a promising signature of early universe tensor perturbations.
However, this promising sign of primordial gravitational waves is partially obscured by a secondary mechanism that can produce B-modes also in the late universe. Indeed the gravitational interaction with large scale structure generates CMB B-modes by distorting primordial E-modes.

As a first approximation, given a convergence field $\kappa= -\frac{1}{2}\nabla^2\phi $ the B mode resulting from the lensing of primordial E mode is:
\be
B^{\mathrm{lens}}(\bl) =  \int \frac{d^2 \bl'}{(2 \pi)^2} W(\bl,\bl') E(\ublu) \kappa(\bl - \bl')
\label{eqn:blens}
\ee
where different modes contributes with a different weight given by:
\be
W(\bl,\bl') = \frac{2 \bl' \cdot (\bl-\bl')}{|\bl-\bl'|^2} \sin(2\varphi_{\bl,\bl'}).
\ee

As usual we define the power spectrum as:
\be
\langle B^{\mathrm{lens}}(\bl) ~ B^{\mathrm{lens}^*}(\tilde{\bl}) \rangle \equiv (2\pi)^2 \delta^D(\bl - \tilde{\bl}) C^{BB,\mathrm{lens}}_l .
\ee

From this we get that the power spectrum of the lensing component of the B-modes:
\be
C_l^{BB,\mathrm{lens}}  = \int \frac{d^2 \bl'}{(2 \pi)^2} W^2 (\bl,\bl')C^{EE}_{l'} C^{\kappa \kappa}_{|\bl-\bl'|}.
\ee

The full B-mode power spectrum measured on the sky is given by a possible primordial component $C_l^{BB,r}$ together with the lensing $C_l^{BB,\mathrm{lens}}$ contribution and the instrumental noise $N_l^{BB}$:
\be
C_l^{BB,\mathrm{full}} = C_l^{BB,r} + C_l^{BB,\mathrm{lens}} + N_l^{BB}.
\ee
Given the current constraints, the lensing component is a significant source of B-modes that, for GW searches purposes correspond to a white noise source of roughly $5 \mu K$-arcmin. This means that it is not only bigger than the biggest allowed GW contribution at scales bigger than several degrees but it is comparable with the decreasing level of instrumental noise. 
For this reason, it is critical to characterize and eventually remove it from the data.
To do so, we need to build a template \refeq{blens} (or its real-space analog) of the actual lensing modes given a measurement of the E-mode field and the lensing potential $\phi$.
While E is measured directly, we ca estimate $\phi$ using "tracers" of the dark matter distribution that creates the potential. 

We will now show how much the delensing efficiency is related to the fidelity of the lensing tracers.
If we have a large scale structure field I($\nver$) that traces the lensing potential responsible for the lensing of the CMB we can build a template of the lensing B mode on the sky by a weighted convolution:
\be
\hat{B}^{\mathrm{lens}}(\bl) = \int \frac{d^2 \bl'}{(2 \pi)^2} W(\bl,\bl') f(\bl,\bl') E^N(\ublu) I(\bl - \bl')
\ee
where $f(\bl,\bl')$ can be determined byminimizing the difference with the true $B^{\mathrm{lens}}(\bl)$ defined in \refeq{blens}.

The residual lensing B mode due to an imperfect knowledge of the true E-mode and $\phi$ will be
\beqn
B^\mathrm{res}(\bl) &=&  B^\mathrm{lens}(\bl) - \hat{B}^\mathrm{lens}(\bl) =  \int \frac{d^2 \bl'}{(2 \pi)^2} W(\bl,\bl') \times \nonumber \\ &&\left( E(\ublu) \kappa(\bl - \bl' ) -  f(\bl,\bl') E^N(\ublu) I(\bl - \bl' ) \right)
\eeqn

The weights $f(\bl,\bl')$ so that the residual lensing B mode power is minimized are:
\be
f(\bl,\bl') = \left(\frac{C^{EE}_{{l'}}}{C^{EE}_{{l'}}+N^{EE}_{{l'}}}\right)  \frac{C^{\kappa I}_{|\bl-\bl'|}}{C^{II}_{|\bl-\bl'|} }
\ee
Notice that the first term consists in an inverse variance filter applied to the measured E-mode and the second minimize the difference between the reconstructed $\phi$ and the CMB lensing potential.

With this choice of $f(\bl,\bl')$ we finally have that the residual power is:
\beqn
\label{eqn:Bres}
C_{{l}}^{BB,\mathrm{res}} &=&   \int \frac{d^2 \bl'}{(2 \pi)^2}  W^2 (\bl,\bl')
 C^{EE}_{l'} C^{\kappa \kappa}_{|\bl-\bl'|}  \\
&\times&  \left[1 - \left(\frac{C^{EE}_{{l'}} }{C^{EE}_{{l'}}+N^{EE}_{{l'}}}\right) \rho^2_{|\bl-\bl'|} \right] \nonumber
\eeqn
with
\be
\rho_l= \frac{\cl^{\kappa I}}{\sqrt{\cl^{\kappa \kappa} \cl^{I I}}}.
\ee

The bigger $\rho_l$ is for a LSS field the more it is correlated with the lensing potential acting on the CMB photons. An higher correlation allows for a better reconstruction of the $\phi^{CMB}$ and, as a consequence, of $B^{lens}$.


\subsection{Multiple tracers of the lensing potential}\label{sec:galcontrib}

Let's now assume that we have n different tracers of the gravitational potentials $I_{i}$ with $i\in \{1,..,n\}$. We can optimally combine them to estimate $\phi$ or, in other word, maximizing the correlation factor $\rho$ with:
\beqn
I 		&=&	\sum_{i}c^{i}I^{i} \nonumber \\
c_{i} 	&=& (C^{-1})_{ij}C^{\kappa I^{j}}
\label{eqn:combined}
\eeqn
where C is the covariance matrix of the LSS tracers.
The residual B-mode power can we computed using \refeq{Bres} using an ``effective'' correlation of these combined tracers with gravitational lensing is:
\be
\rho^{2} = \sum_{i,j}\frac{C^{\kappa i}~(C^{-1})_{ij}~C^{\kappa j}}{C^{\kappa\kappa}}.
\label{eqn:rho-combined}
\ee

Note that the gain in adding a new tracer is not only proportional to its correlation with the CMB lensing but it also depends on how much it is correlated with the already used set of tracers.
\reffig{kernel} show the different kernels as a function of redshift computed using the models and parameters described in \refsec{model}.
Galaxies clustering surveys can only reconstruct the low-z portion of the lensing kernel as can be seen from the LSST,DES and DESI curves.
However given the significant low noise of this measurement they can still help the delensing of the CMB.
Furthermore, the low-redshift tail of the lensing kernel is also the one affected by large scale structures non-linearities. They can indeed be extremely helpful if future CMB surveys will try to remove that contribution instead of attempting to model it.
On the other end, CIB and 21 cm surveys are probing to high redshift structures and independently from the model assumed they show a fairly good overlap with the CMB lensing kernel.


% ===================

\begin{figure}[htbp]
\begin{center}
\includegraphics[scale=1.]{../images/compare_kernel.pdf}
\caption{\textbf{Kernels Comparison}: Comparison of the different kernels as a function of redshift for some of the tracers used in this analysis. The bigger the overlap wit the CMB lensing kernel the better the reconstruction of the lensing potential will be. This will lead to a better delensing. }
\label{fig:kernel}

\end{center}
\end{figure}
% ===================





\section{Lensing potential tracers}
\label{sec:model}
In this section, we describe how the power spectra defined in \refsec{th} are computed and how we characterize different tracers considered in this work.

Large scale structure surveys usually probe the 3D dark matter overdensities as a 2D field projected along the line of sight:

\begin{equation}
\delta^{i}(\nver) = \int_0^{\infty} dz\, W^{i}(z)\delta(\chi(z)\nver,z).
\label{eqn:wkernel}
\end{equation}

where $\delta(\chi(z)\nver,z)$ corresponds to the dark matter overdensity field at a comoving distance $\chi(z)$ and at a redshift $z$ in the angular direction $\nver$.
Using the Limber approximation \cite{limber53} we can compute the power spectra of two large scale structure fields $i,j$ as:
\begin{equation}
C_{\ell}^{ij}= \int_0^{\infty} \frac{dz}{c} \frac{H(z)}{\chi(z)^2}\, W^{i}(z)W^{j}(z)P(k,z).
\label{eqn:wkappa}{}
\end{equation}

In this equation, $H(z)$ is the Hubble factor at redshift $z$, $c$ is the speed of light, $\chi(z)$ is the comoving distance and $P(k,z)$ is the matter power spectrum evaluated at wavenumber $k = \ell/\chi(z)$ and redshift $z$. 
Furthermore $W^{i}(z)$ is the kernel function of the field $i$ defined in \refeq{wkernel}.
We compute these quantities using CAMB and HALOFIT.


\subsection{CMB lensing potential}
\label{sec:kappaCMB}
The first field we compute is the CMB lensing potential.
The lensing kernel $W^{\kappa}$ is:
\begin{equation}
W^{\kappa}(z) = \frac{3\Omega_{\rm m}}{2c}\frac{H_0^2}{H(z)}(1+z)\chi(z)\frac{\chi_*-\chi(z)}{\chi_*},
\end{equation}
%
where $\chi(z)$ is the comoving distance to redshift $z$, $\chi_*$ is the comoving distance to the last-scattering surface at $z_*\simeq 1090$, $\Omega_{\rm m}$ and $H_0$ are the present day values of the Hubble and matter density parameters, respectively.

The CMB lensing potential is \textit{the} field that we need to reconstruct to reverse the effect of large scale structure and delens the CMB.
However, the lensing potential can also be reconstructed from the CMB itself. In that case, we can treat it as a noisy tracer of the true field. Both the CMB lensing fields and its noisy reconstructed counterparts have the same kernel $W^{\kappa}(z)$.
However, when computing the power spectrum of the latter, we need to add a noise component. Given the instrumental noise level and the beam, we can compute the reconstruction noise $N_l^{\kappa \kappa}$, and so
\beq
\cl^{\kappa_\mathrm{rec} \kappa_\mathrm{rec}} = \cl^{\kappa \kappa } +N^{\kappa \kappa}_l
\eeq
In this work, the level of noise is computed assuming an iterative quadratic estimator is used in the reconstruction


\begin{figure}[htbp]
\begin{center}
\includegraphics[scale=1.]{../images/cmb_internal.pdf}
\caption{\textbf{CMB internal delensing:} Correlation factor between CMB internal reconstructed potential and the true lensing potential for different CMB experiments and corresponding noise reconstructions. }
\end{center}
\label{fig:corr-internal}

\end{figure}




\subsection{Galaxies}

We can also write the galaxy overdensity field $g(\nver)$ in a given direction on the sky as a line of sight integral of the matter overdensity:
\begin{equation}
g(\nver) = \int_0^{z_*} dz\, W^{g}(z)\delta(\chi(z)\nver,z),
\end{equation}
where the kernel is
\begin{equation}
\label{eqn:wg}
\begin{split}
W^{g}(z) &= \frac{b(z)\frac{dN}{dz}}{\Bigl(\int dz'\,\frac{dN}{dz'}\Bigr)}.
\end{split}
\end{equation}
Here $\frac{dN}{dz}$ is the number of galaxies as a function of redshift observed by the survey while $b(z)$ is the galaxy bias that connects the amplitude of galaxies overdensity to the one of the dark matter at different redshift.
When computing the auto-spectrum of a galaxies density, a shot noise term needs to be taken into account. 
To do so, we add a constant term to the power spectrum equal to the inverse of the number of galaxies per steradians.
 
Different galaxy surveys in this work are fully characterized by their b(z), $\frac{dN}{dz}$ and the observed galaxies density.
We test the delensing efficiency modeling both current survey like WISE or DES as well as future galaxy probes like DESI and LSST together with 21 cm measurement like SKA.

We defined the WISE redshift distribution following \cite{yan:2013} (see Fig. 4 therein). 
To compute the noise term, we assume that the available sky after masking WISE is around $f_{sky}=0.44$ with 50 million galaxies \cite{ferraro:2015} and that the galaxies density is approximately uniform.
DES is modeled after the DES Science Verification data public release.
For DESI we used the $\frac{dN}{dz}$ in Tab. 2.3 of the DESI Technical Design Report.
LSST is modeled following \cite{schaan:2016} as $\frac{dN}{dz}\propto z^{\alpha}\exp^{-(z/z_{0})^{\beta}}$ with $\alpha = 1.27$, $\beta = 1.02$, and $z_{0}= 0.5$. Furthermore we assume a density of 26 galaxies per arcmin squared.
Finally, we model both the SKA redshift distribution and bias survey following \cite{namikawa:2016a}. 


\subsection{Cosmic infrared Background (CIB)}
\label{sec:cib}
Following \cite{Addison:2011se}, we model the CIB power directly as $C_{\ell}^{{\rm CIB}\mbox{-}{\rm CIB}} = 3500 (l/3000)^{-1.25} {\rm Jy^2 / sr}$.
This model provides an accurate fit to several experimental results.
For the cross-spectra with the CMB lensing or other galaxies tracers $C_{\ell}^{{\rm CIB}\mbox{-}j}$, we use the single-SED model of~\cite{hall10}.
It corresponds to the kernel:
\begin{align}\label{eq:12}
W^{\text{CIB}}(z) = b_c\ \frac{\chi^2(z)}{H(z)(1+z)^2}\ e^{-\frac{(z-z_c)^2}{2\sigma^2_z} } f_{\nu(1+z)},
\end {align}
for
\begin{equation}
f_{\nu} = 
\begin{cases}
\Big( e^{\frac{h\nu}{kT}} - 1 \Big)^{-1} \nu^{\beta+3} & (\nu \leq v^{\prime}) \\ \Big( e^{\frac{h\nu^{\prime}}{kT}} - 1 \Big)^{-1} \nu^{\prime \beta+3} \Big( \frac{\nu}{\nu^{\prime}} \Big)^{-\alpha} & (\nu > v^{\prime})
\end{cases}
\end{equation}

We place the peak of the CIB emissivity at redshift $z_c = 2$ with a broad redshift kernel of width $\sigma_z = 2$. Also $T$ = 34K and $\nu^{\prime} \approx$ 4955 GHz.

%With these assumptions, depending on angular scale, $45-65\%$ of the CIB is correlated with the CMB lensing potential, as shown in \reffig{corrnow}.

\cite{hall:2010,sherwin:2015,ade:2014,hanson:2013,szapudi:2001,0004-637X-567-1-2,planck-collaboration:2014,planck-collaboration:2014a,planck-collaboration:2011,plank-collaboration:2014,boulanger:1996,lewis:2006}


%=================
% FORECAST
%=================

\section{Parameter constraints improvement after delensing}
\label{sec:for}
In this section, we forecast the expected delensing efficiency and the relative importance of galaxies tracers for delensing current and future experiments. We will use a Fisher method and measured the delensign efficiency by the improvement in the constraint of two inflationary parameters: the tensor to scalar ratio $r$ and the tensor tilt $n_{T}$.
We assume a CMB experimental scenario composed of a high-resolution CMB experiments which define the internal reconstruction efficiency, together with a low-noise, low-resolution experiment which B-modes will be delensed and used to constrain the inflationary B-modes.
We focus on three different scenarios: the current stage, the 3G one and the futuristic CMB Stage 4.

\subsection{Fisher method}
In this framework, the error on the cosmological parameters constrained by a B-mode spectrum measurement is given as usual by:
\begin{equation}
\label{eqn:fisher}
\sigma(p)= \left[\sum_l^{\ell_{\rm{max}}} \left(\frac{\partial {C}_{\ell}^{BB,r,n_{t}}}{\partial p}\right)^2/\mathrm{Cov}^2( C_{\ell}^{BB} ) \right]^{-\frac{1}{2}}
\end{equation}
where we consider the parameters $p =\{r,n_{t}\}$. 
We will assume a gaussian covariance:
\beq
\sigma( C_{l}^{BB} )= \sqrt{ \frac{2}{(2 {l}+1) \fsky }} \left( C_l^{BB,r} + C_{{l}}^{BB,\mathrm{lens}}+N_{{l}}^{BB} \right).
\eeq
Here the B-modes noise spectrum is given by:
\beq
N_l^{BB} = \left({\Delta_P}/{T_{\mathrm{CMB}}}\right)^2  e^{ {l^2 \theta_\mathrm{FWHM}^2}/({8 \ln 2})}
\eeq
where $\theta_\mathrm{FWHM}$ is the full half width of the telescope  beam and $\Delta_P$ is the instrumental noise of the experiment.
Even if the covariance structure has known non-Gaussian contributions \cite{motloch:2017,benoit-levy:2012} this approximation is good enough to show the improvement due to delensing.


It can be seen from \refeq{fisher}, that the parameter constraints are inversely proportional to the covariance of the measurement. Since lensing B-mode $C_{{l}}^{BB,\mathrm{lens}}$ are a substantial component of the covariance, removing part of them trough delensing will improve parameter constraints.  

Note that we are fixing the other cosmological parameters. Uncertainties in those will however propagate to uncertainties in $\{r,n_{t}\}$. While the constraints will be slightly optimistic, this has no real impact on the estimate of the improvement due to delensing which is the main focus of this work.
We define the improvement due to delensing as the ration of the constraints before and after delensing: $\alpha_{r} = \sigma^{\rm{del}}(r)/\sigma(r)$ and $\alpha_{n_{t}} = \sigma^{\rm{del}}(n_{t})/\sigma(n_{t})$
%===================
\subsection{Delensing with current CMB and LSS}

% \the\textwidth


Very recently delensing has been proven possible on real data using as tracers both the internal CMB lensing potential reconstruction \cite{carron:2017} as well as the CIB \cite{larsen:2016a,manzotti:2017}.
Here we discuss the improvement due to combining these and other available probes (see \cite{yu:2017} for publicly averrable multi-tracers data-products).

On the CMB side, we will combine a Bicep-Keck like \textit{deep} CMB experiment with a noise equal to \textcolor{red}{$3\mu$K-arcmin} and a beam of \textcolor{red}{30 arcmin}.
The CMB lensing reconstruction will be performed by a higher resolution CMB experiment. We explore two possibilities.
First, we test the efficiency with internal reconstruction performed by Planck \cite{2016A&A...594A..15P} using the CMB lensing noise curves publicly available
\footnote{\url{https://wiki.cosmos.esa.int/planckpla2015/index.php/Specially\_processed\_maps}}.
The second one uses a ground-based experiment with the final noise level of SPT-Pol which correspond to \textcolor{red}{$9.4\mu$K-arcmin} in polarization with a beam equal to \textcolor{red}{1.2 arcmin}. 

Furthermore, we test how much not only CIB maps but also current low redshift galaxies surveys like DES and WISE can improve the delensing efficiency.

The correlation attainable using current experiments is shown in \reffig{corrnow-planck} for Planck and in \reffig{corrnow} for SPT-Pol and.
In general, we observe that CMB lensing reconstruction correlate very well at low $\ell$  and it then falls rapidly because of the the raise of the reconstruction noise.
For the noise levels of Planck, the internal reconstruction is at most $70\%$ correlated at very large scales end falls rapidly to $40\%$ at $\ell=200$. Its correlation is at the level of the one coming from LSS at almost all scales.
On the other end, the CMB lensing reconstruction \cite{} from SPT-Pol will be more than 80\% correlated with the true field at large scales $\ell<300$ where the correlation become comparable with the current LSS surveys.

Following \cite{yu:2017} we cut both the CIB and WISE at $\ell<100$ where they are contaminated by very large Galactic dust residuals.
Optical surveys galaxies like DES are less affected by that systematics and can be used at larger scales.
Furthermore, in \reffig{corrnow} DES contribution to delensing is higher than WISE because the redshift distributions peaks at slightly higher redshift thus overlapping better with lensing.

\begin{figure}[htbp]
\begin{center}
\includegraphics[scale=1.]{../images/actual_scenario_planck.pdf}
\caption{Correlation factor between current galaxies survey and internally reconstructed  $\phi$ CMB lensing potential as a function of the multipole $\ell$. }
\label{fig:corrnow-planck}
\end{center}
\end{figure}


%At large angular scale, the CMB internal reconstruction is clearly the best tracers for delensing. Because of the large level of noise in current experiments, its efficiency became comparable to the CIB one at few degrees ($\ell \simeq 400$).

\begin{figure}[htbp]
\begin{center}
\includegraphics[scale=1.]{../images/actual_scenario.pdf}
\caption{Correlation factor between current galaxies survey and internally reconstructed  $\phi$ CMB lensing potential as a function of the multipole $\ell$. }
\label{fig:corrnow}
\end{center}
\end{figure}



Using this correlation level we can, using \refeq{Bres} compute the residual B-mode power after delensing and test the consequent improvement on parameter constraints with \refeq{fisher}. 
For current surveys we only consider $r$ since the broad beam of the deep BK-like survey does not allow enough leverage to constrain the shape of the B-mode spectra.
We first compute the improvement with a fiducial value of $r_{\rm{fid}}=0$.
The results are summarized in \reftab{current}


\begin{table}
\caption{$\alpha(r)$: Current generation improvements on r due to delensing. Values correspond to ration of the error before and after delensing.}
  \vspace{0.2cm}
  \begin{tabular}{ | c | c | c|}
\hline
Surveys & $\alpha(r=0)$ & $\alpha(r=0.12)$ \\ \hline \hline
WISE & 1.10 & 1.04 \\ \hline
DES & 1.19  & 1.07 \\ \hline
CIB & 1.46  & 1.18\\ \hline
LSS & 1.64  & 1.24\\ \hline
CMB & 1.14  & 1.04 \\ \hline
LSS+CMB & 1.71& 1.26  \\ \hline

  \end{tabular}
  \label{tab:current}
\end{table}



The results are in agreement with \cite{yu:2017,larsen:2016a}, with the CIB being the predominant source of delensing with an improvement of almost 50\%. Planck and DES are essential to reconstruct the largest scales of the lensing potential and as such the play an important role in delensing. The improvement due to WISE is quite marginal, but given the big sky coverage, it might be useful to delens large CMB polarization experiments.
For reference we also test a scenario where GW are present at their highest possible value of $r_{\rm{fid}}=0.12$ \cite{bicep2/keck-collaboration:2015}. As expected the importance of delensing itself is now reduced given that the lensing component constitutes a smaller portion of the total B-modes variance.

As expected the improvement of the internal CMB reconstruction reduces the relative importance of LSS. 
This can be seen in \reftab{current-spt}. 
In particular optical surveys will rapidly lose the role of filling in large-scale modes and they start, as the CIB, to supplement information about lensing at higher multipoles.
 
 
 
 \begin{table}
\caption{$\alpha(r)$: Current generation improvements on r due to delensing. Values correspond to ration of the error before and after delensing.}
  \vspace{0.2cm}
  \begin{tabular}{ | c | c | c|}
\hline
Surveys & $\alpha(r=0)$ & $\alpha(r=0.12)$ \\ \hline \hline
WISE & 1.10 &  \\ \hline
DES & 1.21  & \\ \hline
CIB & 1.52  & \\ \hline
LSS & 1.78  & \\ \hline
CMB & 1.62  &  \\ \hline
LSS+CMB & 2.35 &   \\ \hline

  \end{tabular}
  \label{tab:current-spt}
\end{table}
 
 







\subsection{CMB-S3 Era}
The accuracy of CMB is rapidly improving and the next generation ground based telescopes have been already deployed and are currently taking data.
For the CMB at the beginning of the next decade, we will assume a deep experiment with level of noise of \textcolor{red}{$3\mu$K-arcmin} with a \textcolor{red}{1 arcmin} beam. This is a reasonable level for a combination of the next generation of BK and SPT3G. This level of noise will also be assumed for the internal noise reconstruction.

The correlation attainable using generation 3 experiments is shown in \reffig{corrS3}.
Not only will CMB experiments improved in the near future: possibly D.E.S.I will start taking data. For this reason we add DESI to the previously mentioned LSS tracers. 
DESI will be slightly more efficient than a DES-like surveys because of the slightly higher redshift that it is able to probe.
However the CIB will still play the major role among LSS.

This generation of experiments will start exploring the small scales B-modes thus constraining, even if loosely, the value of $n_{t}$. 
The improvement in cosmological parameters are summarized in \reftab{S3}

  
\begin{figure}[htbp]
\begin{center}
\includegraphics[scale=1.]{../images/S3_scenario.pdf}
\caption{Correlation factor. Same as \reffig{corrnow} but for stage 3 experiments. }
\label{fig:corrS3}
\end{center}
\end{figure}





\begin{table}
\caption{$\alpha$: S3 improvements. Note that the absolute error value of $\sigma(r)$ after delensing for $f_{sky} = $ is  $\sigma(r) =$  }
  \begin{tabular}{ | c | c  |  c |}
\hline
Surveys & $\alpha(r)$ & $\alpha(n_{t})$\\ \hline \hline
DESI & 1.34 & \\ \hline
LSS & 1.71  &\\ \hline
CMB & 1.71  &\\ \hline
\end{tabular}
\label{tab:S3}
\end{table}



\subsection{CMB-S4 Era}

An ambitious program for a generation 4 ground CMB experiment is currently under planning. Moreover satellite experiments have been proposed.
The correlation attainable using current experiments is shown in \reffig{corrs4},
In this very futuristic but reasonable scenario we add to the avaialable large scale structures both 21 cm surveys like SKA and large scale structures like LSST and Euclid.




\begin{figure}[htbp]
\begin{center}
\includegraphics[scale=1.]{../images/S4_scenario.pdf}
\caption{Correlation factor. Same as \reffig{corrnow} but for stage 4 experiments. }
\label{fig:corrs4}

\end{center}
\end{figure}
%=======================================

\begin{table}
\caption{$\alpha$: Gen 4 experiments}
  \begin{tabular}{ | c | c  |  c |}
\hline
Surveys & $\alpha(r)$ & $\alpha(n_{t})$\\ \hline \hline
Euclid & 1.53 & \\ \hline
LSST & 1.65  &\\ \hline
SKA & 4.35  &\\ \hline
CMB S4 & 3.7  &\\ \hline
LSS+CMB & 6.3  &\\ \hline

  \end{tabular}
  \label{tab:S4}
\end{table}


\subsection{Bias uncertainties degradation}
The uncertainties in the theoretical assumptions used to model the galaxies can cause a degradation of the improvement of inflationary constraint of delensed spectra.
In this section we quantify this effect.
We will now marginalize over unknown galaxies parameters but we will use a full dataset of CMB and galaxies data. The idea is that as shown in the low level of noise in galaxies surveys might allow us to internally calibrate them.
We will use a Fisher approach the Fisher matrix is:
\beqn
F_{pq} &=& \sum_{l_{a} =l^{BB}_{\mathrm{min}}}^{l^{BB}_{\mathrm{max}}}  \sum_{l_{b} =l^{BB}_{\mathrm{min}}}^{l^{BB}_{\mathrm{max}}}   \frac{\partial {C}_{l_a}^{BB,\mathrm{del}}}{\partial \theta_p } \left[ \mathrm{Cov}^{BB,BB}\right]_{l_a, l_b} ^{-1} \frac{\partial {C}_{l_b}^{BB,\mathrm{del}}}{\partial \theta_q }\nonumber \\&+&  \sum_j \frac{\frac{\partial {C}_j^{\kappa I}}{\partial \theta_p } \frac{\partial {C}_j^{\kappa I}}{\partial \theta_q }}{(\Delta C_j^{\kappa I})^2}
+\sum_j \frac{\frac{\partial {C}_j^{I I}}{\partial \theta_p } \frac{\partial {C}_j^{I I}}{\partial \theta_q }}{(\Delta C_j^{I  I})^2}
\eeqn

\beq
\alpha_{\mathrm{marginalized}} = \sigma_0(r) / \sigma_{\mathrm{marginalized/delensed}}(r)
\eeq


where the parameter array contains both the tensor to scalar rate $\theta=r$, and the galaxies surveys parameters like the bias $b_i$ or $p_i$ \footnote{In our analysis, since the fiducial value of $r$ is zero,
the derivative of $C_\ell^{\rm BB,res}$ is non-zero if $\theta_i=r$.}.

We compute the derivatives of the power spectra as described


%\begin{table}
%\caption{$\alpha$: improvement on r constraint}
%  \begin{tabular}{ | c | c  | }
%\hline
%Surveys & $\alpha$\\ \hline \hline
%des & 1.34  \\ \hline
%cib & 1.71  \\ \hline
%cmb current & 1.79  \\ \hline
%gals current & 2.03  \\ \hline
%cmb S3 & 2.15  \\ \hline
%gals S3 & 2.16  \\ \hline
%gals S4 & 2.16  \\ \hline
%comb current & 2.66  \\ \hline
%comb S3 & 3.14  \\ \hline
%cmb S4 & 4.36  \\ \hline
%comb S4 & 5.27  \\ \hline
%  \end{tabular}
%\end{table}

\section{Conclusions}
\label{sec:concl}
Delensing or more, in general, the ability to separate the lensing component of the B-mode from a possible primordial inflationary signal is important to fully exploit the capabilities of future experiments.
In this paper, we studied the possible impact of large-scale structure surveys in this important endeavor.
For current experiment, CIB data had already proven to be very efficient in delensing. If no high resolution CMB experiment is available to reconstruct the lensing potential $\phi$ a low-z optical experiment like D.E.S will improve the delensign efficiency by $~10\%$ if combined with Planck lensing and a CIB tracers.
As expected the lower the noise in the CMB experiment the more the delensign efficiency will be predominately come from the internal CMB reconstruction of the lensing potential. Indeed using the CMB itself we reconstruct only the lenses that actually lens the CMB obtained an almost perfect cleaning in the absence of noise. This approach will need a careful study of possible biases coming from using the same source (the CMB) we are delensing to reconstruct the lensing effect itself. This has already been applied to data and studied. However, the fairly good efficiency of galaxies tracers might come in end to cross-check these internal biases. This will probably be needed to confirm a possible detection of gravitation waves if this relies heavily on delensing.

Another possible application of LSS delensing is the removal of the low-z lensing of the CMB that possibly adds the complexity of modeling non linearities in the study of CMB lensing

%=================================================================
\section{Acknowledgments}
We thank S. Dodelson for useful discussion.
This work was partially supported by the Kavli Institute for Cosmological Physics at the University of Chicago through grants NSF PHY-1125897 and an endowment from the Kavli Foundation and its founder Fred Kavli.
%%%=================================================================

\bibliographystyle{plain}
\bibliography{DES_rec_phi_paper,/Users/alessandromanzotti/Work/Astrophysics/latex_bib/cosmobib,/Users/alessandromanzotti/Work/Astrophysics/delensing/sptpol_papers/2016/delens100d/delens100d,/Users/alessandromanzotti/Work/Astrophysics/delensing/sptpol_papers/BIBTEX/spt}

%\bibliography{DES_rec_phi_paper}
\end{document}
